\documentclass[qual, classic, a4paper]{ufbathesis}
\usepackage[utf8]{inputenc}
\usepackage[colorinlistoftodos,prependcaption,textsize=tiny]{todonotes}
\usepackage{float}
\usepackage{microtype}
\usepackage{color}
%\usepackage{xcolor}
\usepackage{indentfirst}
%\usepackage[T1]{fontenc}
\usepackage[alf]{abntex2cite}
\usepackage{acronym}
\usepackage{tikz}
\usepackage{soul}
\usepackage[portuguesekw, portuguese, ruled, linesnumbered]{algorithm2e}
\usepackage{amssymb,amsmath,amsthm}

%\usepackage[symbols,nogroupskip]{glossaries-extra}

%\onehalfspacing
%\newtheorem{theorem}{Theorem}
%\newtheorem{hipotese}{Hipótese}

%%Siglas
\acrodef{HHC}{\textit{Home Health Care}}
\acrodef{HHCP}{\textit{Home Health Care Problem}}
\acrodef{HHCSP}{\textit{Home Health Care Scheduling Problem}}

\acrodef{HHCRSP}{\textit{Home Health Care Routing and Scheduling Problem}}
\acrodef{PERE}{\textit{Problema de Escalonamento e Roteamento de Enfermeiras}}


\acrodef{NSP}{\textit{Nurse Scheduling Problem}}
\acrodef{PEE}{\textit{Problema de Escalonamento de Enfermeiras}}

\acrodef{NRP}{\textit{Nurse Routing Problem}}
\acrodef{PRE}{\textit{Problema de Roteamento de Enfermeira}}

\acrodef{CSP}{\textit{Crew Scheduling Problem}}
\acrodef{PAET}{Problema de Alocação de Equipe técnica}


\acrodef{TSP}{\textit{Traveling Salesman Problem}}
\acrodef{PCV}{Problema do Caixeiro Viajante}

\acrodef{PMCV}{Problema Múltiplo do Caixeiro Viajante}

\acrodef{VRP}{\textit{Vehicle Routing Problem}}
\acrodef{PRV}{\textit{Problema de Roteamento de Veículo}}

\acrodef{VRPTW}{\textit{Vehicle Routing Problem With Time Window}}
\acrodef{PRVJT}{Problema de Roteamento de Veículo com Janela de Tempo}

\acrodef{TSPTW}{\textit{Traveling Salesman Problem With Time Window}}
\acrodef{PCVJT}{Problema do Caixeiro Viajante com Janela de Tempo}

\acrodef{GRASP}{\textit{Greedy Randomized Adaptative Search Procedure}}
\acrodef{AICS}{\textit{Adaptative Iterated Construction Search}}
\acrodef{DLS}{\textit{Dynamic Local Search}}
\acrodef{ILS}{\textit{Iterated Local Search}}
\acrodef{VNS}{\textit{Variable Neighborhood Search}}
\acrodef{VDS}{\textit{Variable Depth Search}}

\acrodef{FESFSUS}{Fundação Estatal Saúde da Família}
\acrodef{SAD}{Serviço de Atenção Domiciliar}

\acrodef{IEEE Xplore}{Instituto de Engenheiros Eletricistas e Eletrônicos}
\acrodef{ACM}{ACM digital Library}
\acrodef{NHS}{National Health Service}

%% Identificacao:

% Universidade
\university{UNIVERSIDADE FEDERAL DA BAHIA}

% Endereco (cidade)
% e.g. \address{Campinas}
\address{Salvador}

% Instituto ou Centro Academico
% e.g. \institute{Centro de Ciencias Exatas e da Natureza}
% Comente se nao se aplicar
\institute{INSTITUTO DE MATEM\'{A}TICA}

% Nome da biblioteca - usado na ficha catalografica
% default: nome da biblioteca do Instituto de Matematica
%\library{BIBLIOTECA REITOR MACÊDO COSTA}

% Programa de pos-graduacao
% e.g. \program{Pos-graduacao em Ciencia da Computacao}
\program{Programa de Pós-Graduação em Ciência da Computação}

% ?rea de titulacao
\majorfield{CIÊNCIA DA COMPUTAÇÃO}

% Titulo da dissertacao/tese
% e.g. \title{Sobre a conjectura $P=NP$}
\title{HEURÍSTICAS PARA O PROBLEMA DE  ESCALONAMENTO E ROTEAMENTO DE EQUIPES DO SERVIÇO DE ATEN\c{C}\~AO DOMICILIAR}

% Data da defesa
% e.g. \date{19 de fevereiro de 2003}
\date{NOVEMBRO DE 2017}

% Autor
% e.g. \author{Jose da Silva}
\author{JÚLIA MADALENA MIRANDA CAMPOS}

% Orientador(a)
% Opcao: [f] - para orientador do sexo feminino
%\adviser[f]{Prof. Dr. TIAGO DE OLIVEIRA JANUARIO}
\adviser{TIAGO DE OLIVEIRA JANUARIO}

% Orientador(a)
% Opcao: [f] - para orientador do sexo feminino
% e.g. \coadviser{Prof. Dr. Pedro Pedreira}
% Comente se nao se aplicar
%\coadviser{NOME DO(DA) CO-ORIENTADOR(A)}

%% Inicio do documento
\begin{document}
% USAR ENQUANTO NAO ESTIVER NA VERSAO FINAL


% Folha de rosto
\frontpage

%%
%% Parte pre-textual
%%
\frontmatter

%\pgcomppresentationpage
% Se seu trabalho for uma tese de doutorado do PGCOMP, use a linha acima
%\pgcomppresentationpage
% Se seu trabalho for uma disserta??o de mestrado do PGCOMP, use a linha acima
% acima e use \presentationpage
\presentationpage
% Se for qualificacao, use \presentationpage

% Ficha catalogrofica
\authorcitationname{Campos, J.} 
\advisercitationname{Januario, T.}

\catalogtype{QUALIFICA\c{C}\~{A}O DE MESTRADO} 


% Dedicatoria
% Comente para ocultar
%\begin{dedicatory}
%DIGITE A DEDICATORIA AQUI
%\end{dedicatory}

% Agradecimentos
% Se preferir, crie um arquivo ?? parte e o inclua via \include{}

\acknowledgements
A todos que de alguma forma ajudaram.

% Epigrafe
% Comente para ocultar
% e.g.
 \begin{epigraph}[]{Lews Carroll}
  Se vo\c{c}\^{e} n\~{a}o sabe pra onde quer ir, qualquer caminho serve.
 \end{epigraph}

% Resumo em Portugues
% Se preferir, crie um arquivo separado e o inclua via \include{}
%\resumo
%DIGITE O RESUMO AQUI
%%%%%%%%%%%%%%%%%%%%%
% Resumo em Português
%%%%%%%%%%%%%%%%%%%%%

\resumo
O Serviço de Atenção Domiciliar, caracteriza-se como uma modalidade de atenção à saúde composta por um conjunto de ações de prevenção, reabilitação e tratamento de doenças, prestadas em domicílio. Esse serviço tem se tornado cada vez mais presente como ação de saúde complementar ou substituto à internação hospitalar, pois oferece uma nova modalidade de atendimento às pessoas com quadro clinico estável que necessitam de cuidados. O roteamento e escalonamento da equipe de internação domiciliar é realizado de forma manual em diversos países, inclusive no Brasil, tornando o processo ineficiente e muitas vezes gerando resultados insatisfatórios. Estima-se que o profissional de atendimento domiciliar passam entre 18\% a 26\% do tempo de trabalho dentro do veículo realizando translados entre os pontos de atendimento. Na literatura, o \textit{Home Health Care Routing Scheduling Problem} (HHCRSP) tem como objetivo construir de forma integrada o roteamento dos veículos da equipe de atendimento domiciliar, assim como, o escalonamento das equipes que serão trasportadas em cada veículo, para que seja possível percorrer todos os locais de visita e atender a um conjunto de pacientes de forma eficiente. Este trabalho tem como objetivo investigar o HHCP e desenvolver uma solução heurística para aumentar o número de atendimentos da equipe de atenção domiciliar, tendo como estudo de caso a Fundação Estatal de Saúde da Família em Salvador.
% Palavras-chave do resumo em Portugues
\begin{keywords}
Escalonamento de Equipes, Roteamento de Veículos, Serviço de Atenção Domiciliar, Heurísticas, Pesquisa Operacional
\end{keywords}

%%%%%%%%%%%%%%%%%%%
% Resumo em Ingles
%%%%%%%%%%%%%%%%%%%

\abstract
The Home Care Service is characterized as a modality of health care composed of a set of actions for prevention, rehabilitation and treatment of diseases, provided at home. This service has become increasingly present as a complementary health action or substitute for hospital admission, as it offers a new modality of care for people with stable clinical conditions that need care. The routing and scheduling of the home hospitalization team is carried out manually in several countries, including Brazil, making the process inefficient and often generating unsatisfactory results. It is estimated that the home care professional spend between 18 \% to 26 \% of the working time inside the vehicle doing transfers between the service points. In the literature, the Home Health Care Routing Scheduling Problem (HHCRSP) aims to build in an integrated way the routing of the vehicles of the home care team, as well as the scheduling of the teams that will be transported in each vehicle, so that it is possible to cover all the places of visit and to attend a set of patients efficiently. This work aims to investigate the HHCP and develop a heuristic solution to increase the number of home care staff, with the State Foundation for Family Health in Salvador as a case study.
% Palavras-chave do resumo em Ingles
\begin{keywords}
Crew Scheduling, Vehicle Routing Problem, Home Health Care, Heuristics, Operational Research.
\end{keywords}


% Palavras-chave do resumo em Portugues
% \begin{keywords}
% DIGITE AS PALAVRAS-CHAVE AQUI
% \end{keywords}

% Resumo em Ingles
% Se preferir, crie um arquivo separado e o inclua via \include{}
%\abstract
% Palavras-chave do resumo em Ingles
% \begin{keywords}
% DIGITE AS PALAVRAS-CHAVE AQUI
% \end{keywords}

% Sumario
% Comente para ocultar
%\tableofcontents

% Lista de figuras
% Comente para ocultar
\listoffigures

% Lista de tabelas
% Comente para ocultar
\listoftables

%%
%% Parte textual
%%
\mainmatter
%\linespread{4}

%capitulos

%%%%%%%%%%%%%%%%%%%
% Sumario / Indice
%%%%%%%%%%%%%%%%%%%

% Comente para ocultar
\tableofcontents

%estilo de numeracao
\pagenumbering{arabic}

% Lista de figuras
% Comente para ocultar
%\listoffigures

% Lista de tabelas
% Comente para ocultar
%\listoftables

%% Parte textual
%\mainmatter



\xchapter{Introdução}{ }

O \ac{SAD}, caracteriza-se como uma modalidade de atenção à saúde composta por um conjunto de ações de prevenção, de reabilitação e de tratamento de doenças prestadas em domicílio.
Esse serviço tem se tornado cada vez mais presente de forma a complementar ou substituir a internação hospitalar, pois oferece uma nova forma de atendimento às pessoas com quadro clinico estável que necessitam de cuidados.
Essa modalidade de atendimento permite maior comodidade aos pacientes, aumentando o conforto e facilitando o apoio familiar, além de reduzir os riscos de contaminação hospitalar e a lotação nos hospitais \cite{Kergosien:2009}.
Por outro lado, o \ac{SAD} também possui algum desafios, tais como: a necessidade do deslocamento do profissional de saúde, o planejamento da escada de trabalho dos profissionais de saúde envolvidos, o aumento de custos para a família, nos casos da necessidade de manter equipamentos elétricos ligados, e a eventual estadia do cuidador ou enfermeiro.%\cite{portaL:2017}.   

Na busca da melhor qualidade de vida da população e na redução de custos hospitalares, o \ac{SAD} tem sido bastante incentivado em diversos países. 
No Brasil, esse serviço teve início na década de 1960, porém seu funcionamento foi regulamentado pelo Sistema Único de Saúde (SUS) na década de 90, a partir da lei n. 8.080, de 19 de Setembro de 1990 \cite{Silva:2010}, chegando a Salvador em 2012, através da \ac{FESFSUS}. 

A \ac{FESFSUS} é um órgão público, sem fins lucrativos, que atua em 69 municípios do Estado da Bahia desde a Lei Complementar Estadual n. 29, de 21/12/2007, tendo iniciado suas atividades em 2009, começando a atuar na Bahia a partir de 16 de Abril de 2012. A fundação possui como uma das suas atribuições fornecer atenção domiciliar, de forma gratuita para os moradores da cidade de Salvador e regiões metropolitanas. Atualmente existem 9 bases e 135 pacientes internados em domicílio, e uma equipe de profissionais composta por: dois médicos, um enfermeiro, quatro técnicos de enfermagem, e um fisioterapeuta, contando também com profissionais de apoio, sendo eles: um assistente social, um nutricionista e um fonoaudiólogo.

A \ac{FESFSUS} fornece serviço de atenção domiciliar a pacientes com médio ou alto grau de complexidade, como por exemplo: pacientes com sequelas de acidente vascular cerebral, cardiopatas, portadores de paralisia infantil, politraumatizados, perfurados por armas de fogo e pacientes em tratamento oncológico.

Apesar do \ac{SAD} já existir há bastante tempo e em diversos países, ainda existem alguns desafios, tais como, o planejamento do escalonamento das equipes de atenção domiciliar e do roteamento dos veículos destinados a conduzir as equipes que irão realizar os atendimentos.

O Problema de Escalonamento e Roteamento de Equipes do Serviço de Atenção Domiciliar, conhecido como \ac{HHCP}, tem como objetivo determinar como as visitas podem ser agendadas, e como as equipes devem ser compostas, de forma a fazer o melhor uso das equipes de profissionais de saúde e atender os pacientes da melhor forma possível~\cite{Bertels:2006} e~\cite{Decerle:2016}.

% O Problema de Escalonamento de Equipe de Atenção Domiciliar, conhecido como \ac{HHCSP} tem como objetivo reduzir os custos da equipe do \ac{SAD}, de forma que o atendimento seja realizado de forma eficiente, sem prejudicar a qualidade do serviço, para  que isso seja possível, é necessário levar em consideração o tempo de atendimento $T[e_{i}, l_{i}]$ e o fato do atendimento ser realizado por um grupo de profissionais com diferentes habilidades, pela preferência dos clientes e pelo meio de transporte utilizado~\cite{Bertels:2006}. 

% O  Problema integrado de Escalonamento e Roteamento da Equipe de Atenção Domiciliar, o \ac{HHCRSP}, tem como objetivo construir de forma integrada o roteamento dos veículos da equipe de atendimento domiciliar, assim como, o escalonamento das equipes que serão trasportadas em cada veículo, para que seja possível percorrer todos os locais de visita e atender a um conjunto de pacientes de forma eficiente \cite{Decerle:2016}.  

Foi verificado na literatura estudada a existência de diversas abordagens heurísticas, técnicas baseadas em inteligência artificial, e técnicas baseadas em métodos exatos para solucionar o problema citado.

\section{Motivação e justificativa}

O roteamento e escalonamento das equipes de internação domiciliar ainda é realizado de forma manual em diversos países, inclusive no Brasil, tornando o processo ineficiente e muitas vezes gerando resultados insatisfatórios~\cite{cheng:98},~\cite{bachouch:2010},~\cite{tozlu:2016} e~\cite{cattafi:2012}.
Estima-se que o profissional de atendimento domiciliar passam entre $18\%$ a $26\%$ do tempo de trabalho dentro do veículo realizando translados entre os pontos de atendimento~\cite{holm:2014}.

Acredita-se que a partir da elaboração de escalas de trabalho e de rotas mais eficientes será possível aumentar a cobertura do serviço e sua visibilidade, permitindo a expansão do atendimento a pacientes com baixa complexidade e o aumento da quantidade de atendimentos a pacientes de média ou alta complexidade.  

Observando as dificuldades encontradas por diversos pesquisadores no momento de realizar o roteamento e o escalonamento do Serviço de Atendimento Domiciliar em vários países do mundo, foi idealizada uma proposta de elaborar um estudo de caso do \ac{SAD} em Salvador, e desenvolver uma solução heurística com o objetivo de aumentar o número de atendimentos da equipe de internação domiciliar. 

% \section{Metodologia}
% A metodologia utilizada neste projeto levará em consideração abordagens heurísticas e técnicas de teoria dos grafos, além de um estudo de caso e pesquisa qualitativa e quantitativa.

\section{Hipótese e objetivos}

Nesta seção será apresentada a hipótese do problema, assim como o objetivo geral e específicos.

\textbf{Hipótese}: \emph{É possível desenvolver uma heurística para o \ac{HHCP} aplicando ao caso específico da equipe de atendimento domiciliar FESFSUS, em Salvador, dessa forma, aumentando produtividade da equipe a partir da redução do tempo dentro do veículo, e auxiliando no aumento da eficiência do atendimento domiciliar em Salvador e região metropolitana,  e contribuindo com a expansão da cobertura do programa, possibilitando o atendimento a pacientes com baixa complexidade.}

Este trabalho tem como objetivo principal desenvolver uma solução heurística para maximizar o número de atendimentos da equipe de atenção domiciliar e aplicar ao projeto FESFSUS em Salvador. 

Buscando alcançar o objetivo principal, temos os seguintes objetivos específicos:
\begin{itemize}
\item Analisar o tempo utilizado pela equipe do \ac{SAD} no percurso entre pontos de atendimentos;
\item Analisar o escalonamento das equipes que serão transportadas em cada veículo;
\item Analisar as rotas de veículos elaboradas pela equipe do \ac{SAD};
\item Analisar heurísticas existentes para os problemas de roteamento e escalonamento do \ac{SAD};
\item Propor uma heurística para o problema de escalonamento e roteamento de equipes do \ac{SAD};
%\item Minimizar os custos do \ac{SAD}
%\item Propor uma heurística para o problema integrado de escalonamento e roteamento de veículos do \ac{SAD};
\end{itemize}


\section{Organização do trabalho}
Este trabalho está organizado da seguinte forma: No capítulo 2 serão apresentados os problemas clássicos de roteamento e de escalonamento; no capítulo 3 serão apresentados problemas de roteamento e escalonamento aplicados à área da saúde; no capítulo 4 é apresenta uma revisão sistemática de literatura, descrevendo as heurísticas existentes; e por fim, no capítulo 5 são apresentados os trabalhos relacionados a proposta da dissertação. 
    
% \begin{figure}[ht]
% \begin{center}
% \begin{tikzpicture}[scale=0.4]
% 	%ponto central
% 	\draw node[draw] at (0, 0) {$0$};
% 	\draw node[draw] at (5,5) {$N+1$};

%     %lado direito x
%     \draw[->, red, dotted, thick] (1.5, -0.5) node[below] {$c_{0,1}$} (0.2, 0) -- (2.8, -1.0) ;
%    	\fill[black] (3,-1) circle (2mm) node[above right] {$v_1$};

%     \draw[->, red, dotted, thick] (4.4, -0.6) node[below] {$c_{1,2}$}  (3.2, -1) -- (5.8, 0);
%     \fill[black] (6,0) circle (2mm) node[above right] {$v_2$};

%     \draw[->, red, dotted, thick] (7.4, -0.1) node[below] {$c_{2,3}$}  (6.2, 0) -- (8.8, 0);
%     \fill[black] (9,0) circle (2mm) node[above right] {$v_3$};   
 
%     \draw[->, red, dotted, thick] (10.4, -0.1) node[below] {$c_{3,4}$}  (9.2, 0) -- (11.8, 0);
%     \fill[black] (12,0) circle (2mm) node[above right] {$v_4$};    

%     \draw[->, red, dotted, thick] (10.8, 2) node[below] {$c_{4,5}$}   (11.8, 0.1) -- (9.3, 3);
%     \fill[black] (9.1, 3.1) circle (2mm) node[above right] {$v_5$};    

%    \draw[->, red, dotted, thick] (6.9, 3) node[below] {$c_{5,6}$}   (9, 3.1) -- (6,2 );
%    \fill[black] (5.8, 2) circle (2mm) node[above right] {$v_6$};    

% 	\draw[->, red, dotted, thick]  (4.1, 2.5) node[below] {$c_{6,7}$}  (5.6, 2) -- (3,2 );
%    \fill[black] (2.8, 2) circle (2mm) node[above right] {$v_7$};  
  	
% 	\draw[->, red, dotted, thick]  (3.2, 4) node[below] {$c_{7,N+1}$}   (2.7, 2) -- (4.8,4.6 );
% \end{tikzpicture}
% \end{center}
% \caption{Grafo de pesos?}
% \end{figure}

\xchapter{Problemas de escalonamento e roteamento}{ }
%Este capítulo apresenta a formulação do problema, as principais definições para o entendimento do mesmo e um resumo de todo o material estudado.}

\presetkeys%
    {todonotes}%
    {inline,backgroundcolor=yellow}{}
% \todo{}


\section{Problema de Alocação de Pessoal}

O Problema de Alocação de Pessoal consiste em atribuir um conjunto de tarefas a um conjunto de pessoas, satisfazendo a um conjunto de restrições \cite{blochiger:2003}.

%O escalonamento de pessoal deve ser realizado de forma que o serviço esteja disponível durante todo o tempo previsto pela organização. Caso o trabalho de uma equipe não interfira ou não dependa do trabalho de outra equipe, então as equipes podem ser escalonadas de forma independente, caso exista alguma relação de dependência, as equipes devem ser escalonadas levando em consideração estas relações \cite{blochiger:2003}.

Seja um conjunto de pessoas $S = \{s_1,s_ 2, ..., s_{|S|}\}$ e um conjunto de tarefas  $F = \{f_1, f_2, ..., f_{|F|}\}$. Uma solução para o Problema de Alocação de Pessoal, também conhecido como \textit{Staff Scheduling}, pode ser representado por uma matriz $M$, na qual cada linha representa uma tarefa que deverá ser executada por uma pessoa $s \in S$ e cada célula $m_{f,s}$, contém o valor referente a atribuição da tarefa $f \in F$ à pessoa $s \in S$.

A Tabela \ref{time_table_block}, representa um exemplo da alocação de um conjunto de tarefas a um conjunto de pessoas de forma que cada tarefa é associada a uma pessoa, sendo que o valor $1$ na célula $m_{fs}$ indica que a pessoa $s$ deve executar a tarefa $f$, caso contrário o valor na célula $m_{fs}$ será 0.

\begin{table}[h]
\centering
\caption{Exemplo de alocação de um conjunto de tarefas a um conjunto de pessoas. \label{time_table_block}} 
\begin{tabular}{r|l|l|l|l}
   & $s_1$ & $s_1$ & $s_3$ & $s_4$ \\ \hline
$f_1$ & 0  & 0  & 1  & 0  \\ \hline
$f_2$ & 1  & 0  & 0  & 0  \\ \hline
$f_3$ & 0  & 0  & 0  & 1  \\ \hline
$f_4$ & 0  & 1  & 0  & 0 
\end{tabular}
\end{table}

% O Problema de Alocação de Pessoal pode ser solucionado a partir da variável de decisão $X_{sf}$ , de forma que se o valor alocado for 1, significa que o item foi alocado na tabela e se for 0, significa que o item não foi alocado. 

% Na equação \ref{alocacao_pessoal}, é representada uma solução para uma instância do problema de alocação de pessoal sendo que $X$ denota o conjunto de todas as variáveis de decisão.

%   \begin{equation}
%   \label{alocacao_pessoal}
%   X_{sf} = 
%   \left \{
%   \begin{array}{cc}
%   1, & \mbox{se a pessoa $s$ foi alocada a atividade $f$} \\
%   0, & \mbox{caso contrario} \\
%   \end{array}
%   \right.
%   \end{equation}

\subsection{Problema de Alocação de Equipe Técnica}
%UMA ABORDAGEM OTIMIZADA PARA O PROBLEMA DE ALOCAÇÃO DE EQUIPES E ESCALONAMENTO DE TAREFAS PARA A OBTENÇÃO DE CRONOGRAMAS EFICIENTES
Seja $k \in K=\{k_1, k_2, ..., k_{|K|}\}$ um conjunto de equipes e $F = \{f_1, f_2, ..., f_{|F|}\}$ um conjunto de tarefas. O Problema de Alocação de Equipe Técnica \ac{PAET}, também conhecido como \ac{CSP} consiste em alocar um conjunto de equipes $K$ a um conjunto de tarefas $F$ \cite{Beasley:1996}. 

Esse problema leva em consideração que todas as equipes são idênticas e estão localizadas no mesmo depósito a partir do qual eles começam e terminam seu dia de trabalho, o número de equipes não pode ser maior do que o número de tarefas que serão executadas e o tempo de execução do conjunto de tarefas não pode exceder o tempo total $T$ \cite{Beasley:1996}.

Para cada tarefa $f$ é associada um custo de execução, uma janela de tempo $[e_f, l_f]$, sendo $e_f$ o tempo de início da execução de cada tarefa $f$ e $l_f$ o tempo final, implicando na duração de tempo $l_f - e_f$, um tempo de viagem $/tau$ e custo $c$ envolvido na viagem da da equipe do depósito para a tarefa $f$ e vice-versa \cite{Beasley:1996}.

Sem perda de generalidade, devemos assumir que as tarefas foram numeradas na ordem ascendente do início. Cada duas tarefas $i$ e $j$, com $j > i$ existe um arco de transição de custo $c_{ij}$, se for possível para a mesma equipe executar a tarefa $i$ e depois executar a tarefa $j$. As tarefas são organizadas de forma a criar um caminho de tarefas que serão executadas pela mesma equipe \cite{Beasley:1996}. 

O objetivo do problema é, encontrar caminhos de custo total mínimo, de modo que cada tarefa seja realizada exatamente uma vez e o tempo de trabalho total envolvido em cada caminho onde, por tempo de trabalho, (significamos o tempo decorrido entre a saída do depósito e a chegada de volta ao depósito) não excede o tempo de trabalho disponível T \cite{Beasley:1996}.

%Uma modelagem para o \ac{CSP} pode ser representada a partir de um grafo $G = (V, A)$, sendo cada tarefa representadas por um vértice, que estão todos ligados entre si por arestas de transição. Existindo dois depósitos mostrados como $0$ para representar o início do caminho e como $N+1$ para representar o fim do caminho. Devido a dimensão do tempo, não existem ciclos no grafo \cite{Beasley:1996}. 
%checar referência
% O objetivo do problema é encontrar os caminhos disjuntos de $K$ vértices no caminho entre $0$ e $N+1$, de modo que todas as tarefas estejam em um caminho, o tempo de trabalho incluído em cada caminho não exceda o tempo total $\tau$, o custo total dos caminhos seja mínimo.
%checar referência
%como pode ser visto na figura \ref{CSP}.

% \begin{figure}[h]
% \centering
% \caption{Exemplo Crew Scheduling Problem com K = 1}
% \centering
% \includegraphics[width=0.8\textwidth]{CSP.png}
% \label{CSP}
% \begin{center}
% Fonte: Elaborada pelo autor
% \end{center}
% \end{figure}

%O objetivo do \ac{CSP} é encontrar todos vértices entre $0$ e $N+1$, tais que todas as tarefas estejam no mesmo caminho, o tempo de trabalho não exceda o tempo total $T$ e o custo total do caminho determinado pelas tarefas executadas seja mínimo. 
% Levando em consideração que cada tarefa pode ser executada apenas uma vez, o custo total associado a cada solução viável é definido pela soma dos custos para realizar cada tarefa $f$, como pode ser visto equação \ref{eqsum}:

% \begin{equation} \label{eqsum}
% 	\sum_{i=1}^{F} c_{f}
%  \end{equation}

% Uma aplicação prática do Crew scheduling problem é o  \textit{Home Care Crew Scheduling Problem}, é um problema no qual uma equipe do \ac{HHCSP} realiza uma série de visitas às casas dos pacientes \cite{rasmussenm:2012}.
%e deve ser encontrado uma solução ótima de forma que todas as visitas sejam feitas no menor tempo possível sem prejudicar a qualidade do atendimento. Este problema foi classificado pelo autor como NP-completo, pois foi verificado a viabilidade em reduzi-lo ao problema do caixeiro viajante, dessa forma, para resolver o problema o autor desenvolveu um algoritmo \textit{branch-and-price}. 

% Uma instância qualquer do problema geral de alocação de equipes pode ser solucionada a partir da variável de decisão $X_{fk}$, tais que, o valor $1$ é associado a uma ligação entre dois vértices, e o valor $0$ determina que não existe uma ligação entre os vértices. Como podemos ver na equação \ref{eqcsp}, que recebe o valor $1$ caso a tarefa $f$ seja associada a equipe $k$ e o valor $0$ caso contrário.

%   \begin{equation} \label{eqcsp}
%   X_{fk} = 
%   \left \{
%   \begin{array}{cc}
%   1, & \mbox{se existe a tarefa $f$ foi alocada a equipe $k$}  \\
%   0, & \mbox{caso contrario} \\
%   \end{array}
%   \right.
%   \end{equation}
  
% A tabela \ref{tarefa_equipe} ilustra um exemplo da solução para o problema de alocação de equipes, no qual cada tarefa $k$ é associada a uma equipe $e_{i}$. 

% \begin{table}[h]
% \centering
% \caption{Tabela: Tarefa X equipes}
% \label{tarefa_equipe}
% \begin{tabular}{l|l|l|l|l}
%    & k1 & k2 & k3 & k4 \\ \hline
% e1 & 0  & 1  & 0  & 0  \\ \hline
% e2 & 0  & 0  & 0  & 1  \\ \hline
% e3 & 0  & 0  & 1  & 0  \\ \hline
% e4 & 1  & 0  & 0  & 0 
% \end{tabular}
% \end{table}

% \subsection{Problema de Alocação de Recursos}

% O problema geral de alocação de recursos consiste em conjunto $R = \{1, 2, ..., r\}$ de recursos que serão escalonados; um custo $c_{r}$ representando o tempo que cada recurso permanecerá alocado, e  um conjunto de processadores  $\Omega = \{1,2, ..., \omega \}$ \cite{ullman:1975}. 
% Como podemos ver na tabela \ref{alocacao_recurso}, no qual o valor é $1$ quando o recurso é alocado no processador $p$ e $0$ quando não é alocado no processador $p$. O objetivo deste problema é executar $k$ recursos dentro do tempo $t$.

% \begin{table}[h]
% \centering
% \caption{Tabela Recurso X Processador \label{alocacao_recurso}}
% \begin{tabular}{l|l|l|l|l}
%    & $\omega1$ & $\omega2$ & $\omega3$ & $\omega4$ \\ \hline
% r1 & 0  & 1  & 0  & 0  \\ \hline
% r2 & 0  & 0  & 0  & 1  \\ \hline
% r3 & 0  & 0  & 1  & 0  \\ \hline
% r4 & 1  & 0  & 0  & 0 
% \end{tabular}
% \end{table}

% O problema de alocação de recursos pode ser solucionado a partir de uma variável de de decisão $X_{r\omega}$, sendo o valor $1$ é atribuído caso o recurso seja associado ao processador, e $0$ se não foi atribuído, como podemos ver na equação \ref{alocacao_recurso}.

%   \begin{equation}
%   \label{alocacao_recurso}
%   X_{rp} = 
%   \left \{
%   \begin{array}{cc}
%   1, & \mbox{ se o recurso $r$ foi alocado ao processador $p$}  \\
%   0, & \mbox{caso contrario} \\
%   \end{array}
%   \right.
%   \end{equation}

\section{Problema do caixeiro viajante}

O Problema do Caixeiro Viajante, também conhecido como \ac{TSP}, é um problema de otimização combinatória no qual, dado um conjunto de cidades e as distâncias estre elas, o objetivo é encontrar o caminho mais curto possível que visite cada cidade exatamente uma vez \cite{goyal:2010}.

Seja um grafo $G = (V,A)$ no qual $V$ é um conjunto de vértices e A é um conjunto de arestas, e seja $C = (c_{ij})$ uma matriz de distância (ou custo), associada com $A$. O \ac{TSP} consiste em determinar um circuito de distância mínima que passa por cada vértice apenas uma vez. Esse circuito é conhecido como um circuito Hamiltoniano \cite{laporte:1992}.

São estudadas diversas variações do \ac{PCV} existindo várias aplicações práticas para o problema, porém neste trabalho serão descritos o Problema Múltiplo do Caixeiro Viajante e do Problema o Caixeiro Viajante com Janela de Tempo. 

O \ac{PMCV}, tem como objetivo determinar um conjunto de rotas de custo mínimo que serão percorridas por um conjunto de Caixeiros Viajantes.

Uma solução para o \ac{PMCV} pode ser representada a partir de um grafo $G = (V,A,W)$ um grafo conectado, onde $V = \{v_1, v_2, ..., v_{|V|}\}$ é um conjunto de cidades, e $A = \{ v_i,v_j: v_i,v_j \in V, i \neq j\}$ é uma aresta com uma matriz custo não negativo $C = {c_{ij}: o peso de (v_i,v_j)}$. $c(v_i, v_j)$ é a distância entre $v_i$ e $v_j$, denotada por $w_{ij}$. d(i) denota o número de arestas conectadas a um vértice $v_i$ no grafo. Um caminho é denotado por uma rota entre dois nós finais com grau 1. Um circuito indica uma rota que começa e termina no mesmo nó. 
\cite{meng:2012}. 

O \ac{PCVJT}, conhecido como \ac{TSPTW}, consiste em encontrar um circuito de custo mínimo, começando e terminando no mesmo depósito e visitando a um conjunto de clientes uma única vez. Cada cliente possui um tempo de serviço definido por uma janela de tempo definindo seu tempo de início e fim do atendimento. As visitas devem ser realizadas respeitando a janela de tempo. O custo de um circuito é a distância total percorrida.~\cite{urrutia:2010}.

Seja $G=(V,A)$ um grafo, onde $V = \{v_0, v_1, ..., n \}$ é um conjunto de n nós, sendo $0$ o depósito e a aresta $A = {(i,j): i,j \in V, i \neq j}$. A janela de tempo é representada por $[e_i, l_i]$ e para cada aresta $(i,j) \in A$ é associado um custo $c_{ij}$ e um tempo de viagem $t_{ij}$ \cite{calvo:2000}.

O Problema do Caixeiro Viajante com o Janela de Tempo é descrito por um conjunto de clientes totalmente interconectados, cada um caracterizado por um intervalo de tempo. Um dos nós representa um depósito, e cada arco possui um tempo e custo de viagem associados. O problema consiste em encontrar um passeio Hamiltoniano de custo mínimo que visita cada nó durante sua janela de tempo. O passeio começa e termina no depósito \cite{calvo:2000}.

\section{Problema de roteamento de veículos}

O problema clássico de roteamento de veículos, conhecido como \ac{VRP}, tem como objetivo encontrar um conjunto de rotas com custo mínimo, começando e terminando a rota no depósito, de modo que seja cumprida a demanda dos clientes. Cada local é visitado uma vez, e cada veículo possui uma capacidade limitada \cite{gold:2008}.

Seja um conjunto de clientes $Q = \{q_1, q_2, ..., q_{|Q|}\}$, que residem em locais $W = \{w_0, w_1, w_2, ..., w_{|W|}\}$ diferentes. Cada par de locais $(i,j)$, onde  $i,j \in W$ e $i \neq j$, é associado a um tempo de translado $\tau$ e uma distância $u_{ij}$. O depósito, local de onde os veículos partem e retornam, é denotado por $w_0$.
Os clientes são atendidos a partir de um depósito com uma frota  homogênea $Z = \{1, 2, ..., z_{|Z|}\}$ de veículos com capacidade uniforme $y$~\cite{gold:2008}.

O Problema de Roteamento de Veículo com Janela de tempo, o \ac{VRPTW}, assim como o \ac{VRP}, consiste em encontrar um conjunto de rotas com custo mínimo, porém cada cliente $q$ possui uma janela de tempo $[e_{q}, l_{q}]$, sendo $e_{q}$ o tempo de início do atendimento e $l_{q}$ o tempo do fim do atendimento~\cite{gold:2008}.
\xchapter{Problemas de escalonamento e roteamento na área da saúde}
{ }

\presetkeys%
    {todonotes}%
    {inline,backgroundcolor=yellow}{}
 %\todo{}
%\section{Escalonamento e roteamento de equipes de enfermagem}


\section{Problema de Escalonamento de Enfermeiras}

%No restante do texto utilizaremos o termo enfermeiras como uma nomenclatura genérica a todos os profissionais de saúde que possam estar envolvidos no Serviço de Atendimento Domiciliar e termo cliente para todas as pessoas que recebam qualquer atendimento de um profissional do \ac{SAD}.

O \ac{PEE} consiste em encontrar uma atribuição de enfermeiras a turnos respeitando a um conjunto de restrições \cite{ioanis:2015}. Seja $E = \{e_1, e_2, \ldots, e_{|E|}\}$ um conjunto de enfermeiras, Seja $D = \{d_1, d_2, ..., d_{|D|}\}$ um conjunto de dias, seja $F = \{ f_1, f_2, ..., F_{|F|} \}$ um conjunto de tarefas e seja $B$ um conjunto de turnos, onde cada elemento de $B$ está mapeado da seguinte forma: ($M$, matutino), ($V$, vespertino), ($N$, noturno) e ($-$, folga). Uma solução para o \ac{PEE} pode ser representada por uma matriz $M_{|E|\times |D|}$, na qual cada célula $m_{e_i,d_j} \in M$ contém a atribuição para o turno $t\in B $ que deverá ser comprido pela enfermeira $e_i$ no dia $d_j$.


A Tabela~\ref{enfermeira_dia} apresenta um exemplo de escalonamento de enfermeiras cujos valores $n_1$, $n_2$ e $n_3$ representam os enfermeiras escalonados no período de sete dias sendo representados pelos valores $d_1$, $d_2$, $d_3$,$d_4$, $d_5$, $d_6$, $d_7$.

\begin{table}[ht]
 \centering
\caption{Exemplo de escalonamento de três enfermeiras no horizonte de sete dias de trabalho \label{enfermeira_dia}}
\begin{tabular}{r|l|l|l|l|l|l|l}
  	   & $d_1$ & $d_2$ & $d_3$ & $d_4$ & $d_5$ & $d_6$ & $d_7$ \\ \hline
 $e_1$ & $M$  & $V$  & $N$  & $M$  & $V$  & $V$  & $-$ \\ \hline
 $e_2$ & $V$  & $-$  & $M$  & $N$  & $N$  & $-$  & $V$\\ \hline
 $e_3$ & $M$  & $M$  & $V$  & $N$  & $-$  & $M$  & $V$
\end{tabular}
\end{table}

No contexto deste trabalho, al\'em de resolver o problema de escalonamento de enfermeiras, tamb\'em devemos nos preocupar com a ordem de atendimento dos pacientes. \'E importante que um determinado paciente que iniciou seu tratamento sob os cuidados da enfermeira $e_i$ continue seu tratamento com a mesma. Sendo assi, \'e necess\'ario integrar o problema de escalonamento ao problema de rotamento, o que aumenta o desafio dessa proposta.

\section{O Problema de Escalonamento e Roteamento de Enfermeiras}

O \ac{PERE} consiste em  encontrar um cronograma, de modo que cada enfermeira visite um conjunto de pacientes, faça uma pausa e finalize as atividades prevista dentro da janela de trabalho. No \ac{PERE}, devemos alocar um conjunto heterogêneo de enfermeiras  que realizam visitas a um conjunto de pacientes  $P = \{ p_1, p_2, \ldots, p_{|P|} \}$, sendo que cada visita realizada a um paciente $p \in P$ por uma enfermeira $e \in E$ possui uma janela de tempo $[c_{p}, f_{p}]$, com $c_{p}$ representando o tempo inicial para chegada e $s_{p}$ representando o tempo de partida do profissional de saúde $p$. A viagem entre as residências de dois pacientes $i$ e $j$  possui o custo $c_{ij}$~\cite{rasmussenm:2012}.   

Cada enfermeira $p \in P$ possui uma janela de tempo $[e_p, l_p]$ que indica o horário de início e fim do seu turno. 

\cite{cheng:98} prop\^os uma modelagem para o \ac{PERE} em um grafo, simples e direcionado $G = (V,A)$. O conjunto de nós $V$, consiste em três conjuntos disjuntos: O conjunto $E$ de enfermeiras, o conjunto $P$ de pacientes e um conjunto $L$ de pausas. O conjunto de arcos $A$, é baseado em uma noção de compatibilidade. Existe uma relação binária entre cada enfermeira e qualquer outro nó. Em outras palavras, a cada enfermeira é permitida ou impedida de visitar um nó em $V$. %Sendo que o gráfico não contém loops nem arcos entre dois nós de enfermeiro ou entre dois nós de almoço.

Cada cliente $q \in Q$ possui o tempo $d_q$, que indica o tempo necessário para fornecer cuidados de saúde ao cliente, e uma janela de tempo $[e_q,l_q]$ no qual $e_q$ é o limite inferior para o tempo de início do atendimento ao cliente $q$ e  $l_q$ é o limite superior do início do atendimento ao paciente $q$. O conjunto de pausas consiste em um nó único para cada profissional de saúde onde cada pausa possui uma duração e uma janela de tempo \cite{cheng:98}.

%O objetivo do \ac{PERE} é maximizar a quantidade de trabalho\footnote{Pense melhor... tem certeza de quem o objetivo seja mesmo maximixar a quantidade de trabalho} minimizando a quantidade de viagens necessária. \cite{cheng:98}\footnote{Verifique novamente no artigo. Certifique-se de que esse artigo nao eh um technical report}

\section{O Problema de Roteamento de Enfermeiras}

Dado um conjunto de locais $W = \{w_0, w_1, w_2, ..., w_{|W|}\}$, sendo que $w_0$ ponto de partida do \ac{SAD}, seja um conjunto $Q = \{q_1, q_2, ..., q_{|Q|}\}$ de clientes e um conjunto de tarefas $F = \{ f_1, f_2, ..., f_{|F|}\}$.  
Uma solução para o \ac{PEE} pode ser representada por um grafo completo $G = (V, A)$, no qual $V = \{0\} \cup Q$ é um conjunto de vértices sendo que cada vértice representa um cliente $q$ que deve ser visitado pela enfermeira $n$, e $A = \{ (i,j): i \in V, j \in V, i \neq j \}$ é o conjunto de arestas, que representa o translado entre os locais $w_i$ e $w_j$, com tempo de viagem $t_{ij}$ associado ao translado de cada enfermeira do local $w_i$ ao local $w_j$ \cite{mansini:2016}.

O \ac{PEE} consiste em determinar um cronograma que atribui a cada enfermeira $n$ quais clientes $q$ devem ser visitados e qual tarefa $f$ deve ser realizada, respeitando um conjunto de restrições\cite{mansini:2016}.

\subsection{Problema de Escalonamento de Equipe de Serviço de Atenção Domiciliar}


%O \ac{HHCSP} tem como objetivo realizar o escalonamento e roteamento da equipe de atenção domiciliar de forma a gerar resultados eficientes~\cite{bachouch:2010}.







\xchapter{Revisão sistemática de literatura}
{ }%Revisão sistemática de literatura}

\section{Protocolo da revisão sistemática de literatura}

A revisão sistemática de literatura é um método de seleção de artigos a partir de um protocolo de busca que tem como objetivos resumir evidências empíricas, identificar lacunas existentes nas pesquisas realizadas até o momento e fornecer estruturas para realizar novas pesquisas. Uma revisão sistemática de literatura é composta por três fases: planejamento da revisão, condução da revisão e análise dos resultados~ \cite{Kitchenham:2007} .

Na fase do planejamento é escolhido qual tema será estudado, define-se a pergunta de pesquisa, que deverá ser respondida após a análise dos artigos estudados, seleciona-se as palavras chave para a criação da string de busca que será utilizada nas bases de dados escolhidas pelo pesquisador e por fim devem ser escolhidos os critérios de inclusão e de exclusão.

Na fase de condução da revisão, as strings de busca criadas deverão ser aplicadas nas bases de dados escolhidas na fase anterior.
Após adquirir o material através da aplicação das strings de busca, os resultados retornados são classificados em dois grupos: artigos  aceitos e artigos rejeitados.

A classificação é realizada através da aplicação dos critérios de inclusão e de exclusão escolhidos na fase anterior.
Quanto à escolha dos critérios de inclusão e de exclusão que serão aplicados no artigo, cabe enfatizar que deve-se seguir a  seguinte regra: se um artigo se encaixar em pelo menos um critério de exclusão, este deverá ser eliminado do grupo de artigos aceitos, mesmo que se encaixe em algum critério de inclusão, e para um artigo pertencer ao conjunto de artigos aceitos devem satisfazer a todos os critério de inclusão e a nenhum critério de exclusão~\cite{Kitchenham:2007}. 

Essa revisão sistemática de literatura tem como propósito: identificar as lacunas existentes no \ac{PERE} e desvendar novas oportunidades de pesquisa e auxiliar pesquisadores na construção de novas soluções para o problema. 

Além dos artigos coletados a partir da chave de busca, outros materiais também foram utilizados para complementar a pesquisa, tais como artigos encontrados a partir de outras bases de dados, partir de outras \textit{strings} de busca, literatura cinza ou livros.

\section{Execução da Revisão Sistemática}

O objetivo dessa revisão sistemática é buscar materiais para investigar a possibilidade de propor um método para solucionar o \ac{PERE}, de forma que seja possível aumentar  o número de clientes atendidos e reduzir o tempo de transporte da equipe de profissionais de saúde. 
A partir do objetivo da pesquisa foi encontrado a seguinte questão de pesquisa:~\emph{É possível maximizar o número de visitas da equipe de atendimento domiciliar a partir da aplicação de soluções heurísticas?}

Para responder a questão de pesquisa foram examinados trabalhos aplicados ao \ac{PERE}.
%Um dos resultados esperados ao final da pesquisa é a identificação de lacunas existentes no \ac{PERE} e catalogar métodos existentes.
Para a elaboração da pesquisa foram utilizadas as seguintes palavras chave: \textit{``home health care''}, \textit{``home care'', ``nurse scheduling''} e \textit{``nurse routing''}, organizadas na seguinte string de busca: \textit{``(``home health care'' OR ``home care'') AND (``routing OR scheduling'')''} e \textit{``nurse AND (routing OR scheduling)''}. 

A Pesquisa foi restrita a artigos escritos na língua inglesa e portuguesa, selecionados a partir seguintes bases de pesquisa: \textit{Science Direct, Springer}, \textit{IEEE Xplore}, \textit{Scopus} e \textit{ACM Digital Library}. Além dos artigos selecionados a partir da strings de busca nas bases citadas, a pesquisa conta com outros materiais, tais como livros, materiais retirados da literatura cinza e artigos adquiridos de forma \textit{ad-oc}.

A pesquisa realizada a partir da string de busca retornou 151 resultados distribuídos da seguinte forma: 73 artigos da \textit{Science Direct}, 50 artigos do \textit{IEEE Xplore}, 16 artigos da \textit{Springer}, 7 artigos da \textit{Scopus} e 5 artigos da \textit{ACM digital Library}.% como pode ser observado na Figura~\ref{base_de_dados_de_origem}.

%Colocar figura nos slides
% \begin{figure}[H]
% \begin{center}
% \includegraphics[width=0.6 \textwidth]{contagem_base_de_dados_de_origem.png}
% \caption{Contagem de artigos por bases de dados \label{base_de_dados_de_origem}}
% \end{center}
% \end{figure}

Após a realização da busca dos artigos nas bases de dados especificadas, os artigos encontrados foram selecionados segundo os seguintes critérios de inclusão e de exclusão.

\textbf{Critérios de inclusão:}
\begin{itemize}
\item Artigos com foco em problemas de escalonamento de profissionais de saúde;
\item Artigos com foco em problemas roteamento de veículos das equipes de profissionais de saúde;
\item Artigos publicados em revistas ou conferências;

\end{itemize}

\textbf{Critérios de exclusão:}
\begin{itemize}
\item Artigos que não estão em inglês ou português;
\item Artigos inacessíveis ou indisponíveis;
\item Artigos duplicados;
\item Revisões de literatura.
\end{itemize}

Após a aplicação dos critérios de inclusão e de exclusão nos 151 artigos coletados, 38 artigos foram aceitos por atenderem a todos os critérios de inclusão e 113 artigos foram rejeitados por se encaixarem em pelo menos um critério de exclusão. %Como pode ser observado na figura \ref{status}

%colocar figura nos slides
% \begin{figure}[H]
% \begin{center}
% \includegraphics[width=0.8 \textwidth]{status.png}
% \caption{Contagem bases de dados \label{status}}
% \end{center}
% \end{figure}

\section{Comentários sobre os artigos selecionados}

Foi elaborada por \cite{santos:2015} uma solução para o \ac{PEE} baseada em Programação por Restrições Ponderadas na qual dado uma série de restrições, visa minimizar o peso total de todas as restrições. Para a elaboração da solução os autores utilizaram uma variante do \ac{PEE} como um problema de vários estágios com 4 a 8 semanas, onde uma semana é considerada um estágio separado. O objetivo da solução elaborada é atribuir um número fixo de enfermeiros a um conjunto de turnos. As restrições que devem ser satisfeitas pela solução elaborada pelos autoes são divididas em fortes e fracas. Após e execução dos testes os autores tiveram como resultado que, após cerca de 1000 passos, as restrições fortes foram satisfeitas, as restrições fracas foram sendo resolvidas durante a fase de busca local.

\cite{mansini:2016} propôs uma técnica de programação matemática para solucionar o Problema de Roteamento de Enfermeiras, modelando o problema como uma variante do \textit{Multi-Vehicle Traveling Purchaser Problem} (MVTPP-PIC), elaborando uma solução a partir da abordagem \textit{Branch and Price}. Para realizar os testes computacionais, os autores geraram instâncias 10 novas instâncias para testar os algoritmos, adotando parcialmente o método de geração proposto para o MVTPP-PIC um artigo publicado por eles mesmos no ano anterior, encontrando uma solução ótima em três de  dez instâncias.

\cite{bachouch:2010} desenvolveram uma ferramenta de otimização e propuseram uma abordagem baseada em Programação Linear Inteira  para enfermeiros em um escritório francês HHC, com o propósito de equilibrar a carga de trabalho, minimizando a diferença entre o limite superior e inferior da carga de trabalho. Os autores realizaram 60 experimentos, baseados em 10 conjuntos de 10, 15 e 20 tarefas para 3 e 5 enfermeiras, gerados aleatoriamente. Após a realização dos testes, os autores conseguiram equilibrar a carga de trabalho entre os profissionais de saúde.

Foi desenvolvido uma solução heurística para solucionar o \textit{Crew Constrained Home Care Routing Problem with Time Windows} por \cite{tozlu:2016}, descrevendo o problema a partir de um modelo de Programação Linear Inteira Mista, com o objetivo de minimizar a distância total viajada pelas equipes. Para a realização dos testes computacionais foi gerado aleatoriamente um conjunto de instâncias que foram  resolvidas a partir do \textit{Variable Neighbourhood Search}. Os autores conseguiram alcançar o objetivo em 175 instâncias das 192 instâncias testadas.

dFoi proposto por  \cite{trabelsi:2012} um modelo de escalonamento baseado em Programação Linear Interira para  equilibrar a carga de trabalho dos enfermeiros do \ac{SAD},  minimizando a diferença entre o limite superior e inferior da carga de trabalho. O modelo foi testado para diferentes tamanhos de problemas: de 4 a 15 pacientes para 2, 3 e 4 enfermeiras, sendo apresentados apenas resultados obtidos para 7, 10 e 15 pacientes e 2, 3 e 4 enfermeiros.

Uma abordagem baseada em Programação por Restrições para o \ac{PERE} foi elaborada por \cite{cattafi:2012}, tendo como estudo de caso a realidade enfrentada pela cidade de Ferrara na Itália, onde o problema é resolvido manualmente. O autor formalizou o problema através da interação com a equipe de enfermagem do hospital da cidade, tendo como objetivo minimizar a carga horária máxima semanal. Ao final do experimento, os autores chegaram a conclusão de que a Programação Lógica pode ser eficaz para resolver problemas da vida real.

Foi proposto por \cite{Decerle:2016} uma solução baseada em Programação Linear Inteira Mista com o objetivo de minimizar os custos relacionados ao transporte e as horas trabalhadas pela equipe do \ac{SAD}. Para a execução de testes computacionais, foram utilizadas instâncias de dois escritórios do Serviço de Atenção Domiciliar.

Uma solução heurística proposta para o \ac{HHCP} por \cite{Bertels:2006} é baseada na combinação de Programação Linear Inteira, Programação por Restrições e meta-heurísticas. As instâncias utilizadas para gerar a solução foram obtidas a partir de 10 cenários sintéticos, contendo entre 20 e 50 profissionais e entre 111 e 326 trabalhos.

Foi formulada uma nova solução para o \ac{PEE} por \cite{baskaran:2014} utilizando Programação Linear Inteira, simplificando o problema através da Granulação de Informações. Tendo como objetivo principal desenvolver um modelo de agendamento, que satisfaça todas as restrições difíceis e que a preferência da enfermeira em relação aos dias de descanso seja maximizada. Para a realização de experimentos os dados foram obtidos dados reais, consistindo em oito enfermeiros, divididos em três turnos e um período de repouso.

Foi desenvolvida um solução baseada em Programação Inteira Binária para Problema de Escalonamento de Enfermeira, por \cite{Zen-El-Din:2012}, tendo como principal objetivo automatizar o escalonamento das enfermeiras do Hospital do Cairo. Os autores realizaram um estudo de caso  na unidade de terapia intensiva no Hospital do Cairo, sendo que os dados foram coletados de enfermeiros-chefe, supervisores, enfermeiros e assistentes.

Foi elaborado por \cite{burke:2010} um modelo hibrido que combina Programação Linear Inteira e Busca em Vizinhança para lidar com o \ac{PEE}. A abordagem hibrida proposta pelos autores produz uma melhora de 15,5\% quando comparada a outras abordagens semelhantes.

Foi proposta por \cite{ohki:2008} uma solução para o \ac{PEE}, baseado em uma técnica de ajuste de peso de penalidade executando Algoritmos Genéticos Cooperativos, levando em consideração a realidade de um hospital. Os autores propõem uma técnica de ajuste de peso quando a função de penalidade estagna no mínimo local. Os experimentos foram realizados em uma base de dados real, com 23 enfermeiras. Os autores tiveram como resultado uma solução dez vezes mais rápida quando comparada com a técnica convencional. 

\cite{ohki:2008} propôs um Algoritmo Genético Cooperativo Paralelo para solucionar o Problema de Alocação de Enfermeiras, levando em consideração a realidade de um hospital. Os autores tem como objetivo otimizar o cronograma de alocação de enfermeiras usando apenas o operador de crossover. Chegando a conclusão de que o processamento paralelo oferece a solução mais adequada através de a otimização entre 400.000 gerações de 100.000 gerações.

Na modelagem multi objetiva, existem vários objetivos que estão em conflito entre si, e existem algumas restrições rígidas que devem ser satisfeitas em qualquer solução. Foi apresentado por\cite{ahmet:2009} uma abordagem baseada em Algoritmos Genéticos Multi-Objetivos com memória externa para resolver o Problema de Escalonamento de Enfermeiras, tendo como principal objetivo minimizar a quantidade de dias sucessivos trabalhados pelas enfermeiras. Os autores estudaram soluções para o \ac{PEE} utilizando Algoritmos Genéticos Multi-Objetivos com e sem memória externa, obtendo melhores soluções quando aplicado a técnica com memória externa. 

\cite{mozab:2007} descreve  heurísticas construtivas, além de várias versões de algoritmos genéticos baseados em codificações específicas e operadores para problemas de sequenciamento aplicados ao Problema de Escalonamento de Enfermeira. Nos algoritmos genéticos descritos, cada indivíduo na população está associado a um par de cromossomos, representando permutações de tarefas e enfermeiros. Essas permutações são usadas como entrada para um procedimento que gera listas. A adequação dos indivíduos é dada pela semelhança entre a lista gerada a partir das permutações e a atual. Os autores desenvolveram várias versões do algoritmo genético, cuja diferença estava na codificação das permutações e nos operadores genéticos utilizados para cada codificação. Estas heurísticas foram testadas com 68 instâncias reais,  cirando aleatoriamente ausências em duas unidades de hospitais públicos em Portugal, envolvendo 19 e 32 enfermeiras, respectivamente.

\cite{nguyen:2016} apresenta uma solução meta-heurística com base em Algoritmos Genéticos para solucionar os problemas de arquivamento, atribuição, roteamento e agendamento em planejamento de mão-de-obra multi-período sob incerteza na disponibilidade de enfermeiras. Os testes computacionais foram realizados com instâncias baseadas em duas bases de dados reais de uma empresa do \ac{SAD} que opera na Suíça. Nos dois conjuntos de dados, o número de pacientes é de 190, o número de requisitos médicos é de 760, o número de enfermeiros oficiais é de 15, o número de tipos de desempenhos é de 22, o número de turnos é de 3, o número de dias é de 7. Os resultados obtidos mostram que, abordando todos os problemas de otimização em um modelo integrado e resolvendo o modelo de forma unificada, diminuem substancialmente o custo operacional. Além disso, mostra que o algoritmo proposto: funciona melhor com a estratégia de substituir as soluções aleatoriamente em seu processo de evolução; é computacionalmente eficiente para lidar com grandes instâncias e para fornecer uma solução que seja confiável contra a incerteza na disponibilidade de enfermeiros para a força de trabalho semanal planejamento; pode ser uma ferramenta para avaliar o \textit{trade off} entre a robustez e o custo operacional de uma solução.

\cite{tsai:2009} desenvolveram uma modelagem matemática em dois estágios para o \ac{PEE}. Em sua abordagem, na primeira etapa organiza-se o horário de trabalho e de folga das enfermeiras,  sendo utilizado uma abordagem baseada em Algoritmos Genéticos para resolver os horários. No segundo estágio, o cronograma das enfermeiras foi organizado e foi utilizado um  Algoritmo Genético com o objetivo de encontrar o horário ideal. O autor realizou um estudo de caso empírico, com 15 equipes de enfermagem de um hospital em Taiwan, tendo como principal resultado que o modelo elaborado é altamente adaptável a diferentes casos.

 \cite{luna:2013} propuseram uma heurística para o Problema de Escalonamento de Equipe técnica baseada em Algoritmos Evolutivos, utilizando quatro instâncias reais, fornecidas por uma empresa privada. Os resultados mostram,  que o algoritmo proposto retornou resultados eficientes a partir da base de dados real utilizada e  que a paralelização proposta permite ao algoritmo abordar adequadamente instâncias com mais de 10000 instâncias. 
 
\cite{maenhout:2011} propôs uma mete heurística evolutiva para resolver o \ac{PEE}, tendo como objetivo maximizar o número de tarefas realizadas, dessa forma evitando desvios no cronograma. Para a realização dos testes computacionais foram gerados dados aleatoriamente. Os resultados computacionais mostram que o procedimento produz um equilíbrio certo entre um mecanismo de diversificação bem-realizado e três mecanismos complementares de intensificação. 

\cite{constantino:2011} elaborou uma abordagem heurística para o \ac{PEE}, baseada em Satisfação de Preferência Balanceada. O algoritmo elaborado pelos autores possui duas fases. Os testes computacionais foram realizados com instâncias do \textit{Standard Benchmark Bataset} e a partir dos experimentos os autores concluíram que o método utilizado gerou resultados efetivos e eficientes para solucionar o problema.

O algoritmo de Busca Harmônica  tem sido utilizado para solucionar problemas de otimização, este método funciona a partir da imitação do processo de improvisação musical.  Foi elaborada uma abordagem meta-heurística baseada em Busca Harmônica para o \ac{PEE}, por\cite{awadallah:2011}. Para realizar os testes computacionais foram utilizados dados da competição Internacional de Escalonamento de Enfermeiras.

\cite{wu:2015} elaborou uma formulação matemática para o Problema de Escalonamento de Enfermeira e propôs uma solução baseada em Enxame de Partículas, realizando um estudo de caso em um hospital em Taiwan, com objetivo de minimizar a quantidade de desvios de cada turno de trabalho e dia de folga. Os testes foram realizados em três conjuntos de dados reais do hospital estudado, produzindo soluções ótimas nos três problemas testados.

Foi apresentado por \cite{lu:2012} um método baseado em \textit{Adaptive Neighborhood Search} para resolver o \ac{PEE}.  Os resultados computacionais avaliados nos três conjuntos de 60 instâncias da competição mostram que o algoritmo utilizado melhora os resultados mais conhecidos por 12 instâncias ao combinar os melhores limites para outras 39 instâncias.

\cite{altamirano:2010} apresentou uma abordagem baseada em Enxame de Partículas para solucionai o Problema de Escalonamento de Enfermeira Anestesiologista em um hospital público francês, com o objetivo de maximizar a equidade do cronograma. Os autores compararam a solução elaborada com técnicas de Programação Linear e Programação por Restrição, encontrando a solução ótima em menos tempo.

Baseado na teoria Fuzzy, \cite{mutingi:2013}  elaborou uma heurística para o escalonamento de Equipe de Atenção Domiciliar, utilizando o método de Enxame de Partículas Fuzzy, tendo como objetivo minimizar o desequilíbrio da carga de trabalho, evitando viagens de longa distância para pacientes e violação das janelas de tempo dos pacientes. Os testes foram realizados a partir de dados obtidos em \cite{bachouch:2010}. 

Foi elaborado um modelo baseado em Lógica Fuzzy para solucionar o Problema de Escalonamento de Enfermeira por \cite{topaloglu:2010}, utilizando dados do mundo real para validar os testes computacionais. 

Um modelo de Programação Inteira Multi Objetivo para o Problema de Escalonamento de Enfermeira foi proposto por \cite{cetin:2015}, em seguida os autores elaboram uma abordagem baseada em Lógica Fuzzy e aplicam a um estudo de cado em um hospital em Konya, na Turquia.

Foi elaborado por \cite{gutjahra:2007} uma heurística baseada em Colônia de Formiga para resolver o \ac{NSP} no \textit{Vienna Hospital Compound} na Áustria. Os autores compararam o a solução elaborada com Algorítimos Gulosos. Após realizar experimentos com 50 dados diferentes, foi observado que o algoritmo proposto alcança melhorias significativas em comparação a um algoritmo de abordagem gulosa.

Em busca de encontrar a solução para o Problema de Roteamento e Escalonamento da Equipe de Atenção Domiciliar, \cite{trautsamwieser:2014} elaboraram uma heurística  \textit{Branch-Price-and-Cut} usando as soluções de uma abordagem de solução de pesquisa de vizinhança variável como limites superiores para o Problema. Em testes computacionais o autor testou sua abordagem com instâncias baseadas em mais de nove enfermeiras, 45 clientes e 203 visitas durante a semana.

%\section{Análise descritiva dos resultados encontrados}

%Na seção anterior foi realizada uma revisão das soluções heurísticas encontradas a apartir da revisão sistemática de literatura, nessa parte serão descritas com mais detalhes algumas soluções heurísticas mais utilizadas pelos autores.

% \subsection{Técnicas de inteligência artificial}

% Vários autores utilizaram técnicas de inteligência artificial para elaborar heurísticas para o \ac{HHCSP} e o \ac{HHCRSP}, podendo citar as técnicas seguintes como as mais utilizadas.

%\subsection{Lógica Fuzzy}

% \
% \linebreak[4]

%A teoria de conjuntos difusos, do inglês \textit{fuzzy}, trabalha com modelos de conjunto cujo seus elementos possuem um determinado grau de pertinência a este conjunto, não aceitando valores booleanos, como acontece em conjuntos precisos, ou seja, que não são difusos \cite{mutingi:2013}.

%Para melhor entendimento sobre a teoria de conjuntos difusos, \citeonline{mutingi:2013} descreve a seguinte diferença entre os conjuntos:

%Seja um conjunto universo $X$ composto por elementos $x$, e o subconjunto $A$ do conjunto universo $X$, tal que $A \subseteq X$. 

%O elemento $x$ é um membro de um conjunto preciso se é definido pela função transformação $\mu_{A}$ a partir de $X$ em {0,1}, desde que:

% \begin{equation}
%  \mu_{A}(x) = 
%  \left \{
%  \begin{array}{cc}
%  1, & se \ x \ \in \ A  \\
%  0, & se \ x \ \notin \ A \\
%  \end{array}
%  \right.
%  \end{equation} 

%Em um conjunto difuso, o elemento $x \in A$  é definido por $\mu_{A}(x) \in [0,1]$, cujo cada elemento em X possui o valor dentro do intervalo [0,1] sendo que quanto mais próximo o valor de $\mu_{A}(x)$ está de $1$, maior é o grau de pertinência do elemento $x$ em A, e quanto mais próximo de $0$, menor o seu grau de pertinência.


%\subsection{Algoritmos Genéticos \hfill}
% \
% \linebreak[4]

%Algoritmos genéticos pertencem a uma subclasse de algoritmos evolutivos e possuem seu funcionamento baseado na teoria evolutiva da biologia proposta por Darwin. Nas ultimas décadas os algoritmos evolutivos tem sido bastante utilizados para resolver problemas de otimização.~\cite{malhotra:2011} 

%Os algoritmos genéticos contém um cromossomo, um gene, conjunto populacional, um \textit{fitness} e função \textit{fitness}, reprodução, mutação e seleção.

%O funcionamento dos algoritmos genéticos começam com um conjunto solução representado por um conjunto de cromossomos, chamado população, na qual cada cromossomo representa um individuo. As soluções são selecionadas de acordo com sua aptidão para gerar novas soluções, chamadas descendência, cada solução gerada para uma população é utilizada para uma nova população, possivelmente melhor do que a população anterior. 

%O processo utilizado para gerar uma população é repetido até que as condições pré definidas sejam satisfeitas, como pode ser visto no algoritmo~\ref{genetico}.

%\begin{algorithm}[ht]
% \Entrada{$P$} //população inicial \\
% \Saida{individuo que satisfaz a condição pré definida}
%   \Inicio{
%	Selecione uma subpopulacao $P'$
%     \While { $X$ não é satisfeito }{
%        escolha $S_{i}, S_{j} \in P'$ \\
%        $S_{k} \longleftarrow CRUZAMENTO(S_{i}, S_{j})$ \\
%      \eIf{$f(S_{i} \geq f(A_{j})$}{
%        	$S_{melhor} \longleftarrow S_{i}$\\
%        	}{$S_{melhor} \longleftarrow S_{j}$\\} 
%       \eIf{$f(S_{melhor}) \geq f(S_{k})$}{
%       	  $SUBSITUI(S_{melhor}, P)$\\
%        }{não substitui}  
%        $S_{k} \longleftarrow MUTACAO(S_{k})$ \\
%        }
%      \Retorna{$S_{k}$} \\
%   }
% \caption{Algoritmo genético \label{genetico}}
%\end{algorithm}

%O pseudocódigo acima ilustra o funcionamento do Algoritmo Genético, no qual a entrada é uma população P e a saída é um indivíduo $S$. Primeiramente é selecionada uma população inicial $P$ que irá gerar uma nova população. Até que o critério de parada seja satisfeitos, é gerada uma nova população $P'$, subconjunto da população $P$, da seguinte forma: São escolhidos aleatoriamente dois cromossomos $S_{i}$ e $S_{j}$ em cada subpopulação $P'$, estes cromossomos são recombinados a partir do cruzamento dos mesmos, levando a formação de um novo cromossomo $S_{k}$. O cromossomo $S$, representa o valor da função objetivo, dessa forma, quanto menor o valor, melhor será a adequação do indivíduo representado pelo cromossomo. Após a comparação do cromossomo filho com os cromossomos pais, é realizado o procedimento de mutação, gerando uma nova descendência.

%A cada iteração do algoritmo é testado se o indivíduo gerado satisfaz a condição $X$ pré definida, caso as condições sejam satisfeitas, o algoritmo para e retorna o indivíduo, caso contrário ele segue até que alcance alguma outra condição de parada $X$.

%\subsection{Memetic Algorithm}
% \
% \linebreak[4]

%\textit{Memétic Algorithm} são metaheurísticas baseadas em população, compostas por uma estrutura evolutiva e um conjunto de algoritmos de busca local. 
%Apesar de ser utilizado para resolver problemas de otimização, apesar de sua utilização o \textit{Memétic Algorithm} não é proposto como um algoritmo de otimização, mas como uma ampla classe de algoritmos inspirados pela difusão das ideias ideias e compostas por múltiplos operadores existentes~\cite{neri:2012}.  
 
%O \textit{Memétic Algorithm} visa a convergência de uma população para uma solução ótima, como segue:
%Primeiro é realizada a seleção aleatória das soluções candidatas iniciais, chamada de pais, que serão utilizadas para a criação de uma nova solução. Essa seleção é realizada levando em consideração o desempenho das soluções candidatas, quanto melhor o desempenho, maiores são as chances de seleção das soluções.
%A segunda etapa do algoritmo consiste em combinar os pais e criar uma nova solução candidata, mais próxima do resultado esperado.
%Após a geração da nova população, ocorre a fase de mutação, na qual é realizada uma melhoria no novo resultado gerado.   
%A ultima etapa decide se a nova população gerada deve se tornar membro da população e qual solução existente deve ser substituída.

%\subsection{Colônia de formiga}
% \
% \linebreak[4]

%O algoritmo de Colônia de Formigas, possui seu funcionamento baseado no comportamento das formigas na natureza, particularmente no processo realizado por estas ao procurar comida. Quando as formigas vão em busca de comida, elas partem aleatoriamente até encontrar a comida, ao encontrar a comida a formiga retorna para o seu ninho deixando uma trilha de ferormônios, que será seguida pelas outras formigas~\cite{blum:2005}.

%O algoritmo de Colônia de Formigas é descrito como um grafo G = (V,E), onde V consiste em dois nós nomeados $v_{s}$, representando o formigueiro e $v_{d}$, representando a comida; e $E$ representa duas ligações, $e_{1}$ e $e_{2}$ o caminho entre o formigueiro, representado por $v_{s}$ e a comida, representada por $v_{d}$. Sendo que $e_{1}$ representa o caminho mais curto e $e_{2}$ representa o caminho longo. 

%A representação da trilha de ferormônio é modelada da seguinte forma: É introduzido um ferormônio artificial $t_{i}$ para cada duas ligações $e_{i}$, sendo i=1,2; indicando a força da trilha deixada. Por fim, são introduzida formigas artificiais, representadas por $n_{a}$. Para criar o caminho de cada formiga do formigueiro até a comida, ou seja de $v_{s}$ para $v_{d}$, cada formiga que parte do ponto $v_{s}$ é escolhida com probabilidade: 

%\begin{equation}
%P{i} = {t_{i}\over\displaystyle {t_{1}+t_{2} } }, \ para\ i = 1,2
%\end{equation}

%entre os caminhos $e_{1}$ e $e_{2}$ para alcançar a comida $v_{d}$. Se $t_{1} > t_{2}$, a probabilidade do caminho 1 ser escolhido é maior.
%Para criar o caminho de volta para o formigueiro, ou seja, o caminho de $v_{d}$ para $v_{s}$ é utilizado o mesmo caminho criado de $v_{s}$ para $v_{d}$~\cite{blum:2005}.

%\subsection{Algoritmo Guloso}

%Algoritmos Gulosos sempre escolhem a melhor solução para o momento partindo do ótimo local em busca do ótimo global. Apesar deste algoritmo encontrar a solução ótima para vários problemas, nem sempre é possível alcançar a solução ótima a partir da execução deste algoritmo~\cite{Cormen:2009}. 

%Em um Algoritmo Guloso, a cada iteração, um novo elemento do conjunto solução é incorporado a construção da solução parcial até que a solução completa seja obtida. A seleção do próximo elemento que será incorporado é determinado a partir da função gulosa de avaliação. esta função gulosa representa o aumento incremental na função custo para a incorporação deste elemento na solução parcial em construção~\cite{resende:2014}. O critério para estabelecer qual elemento será selecionado é o do elemento menos custoso, como podemos ver no algoritmo~\ref{guloso}.

%\begin{algorithm}[ht]
% \Entrada{$C \neq \emptyset$} //conjunto de candidatos \\
% \Saida{melhor solução segundo o critério guloso}
%   \Inicio{
%   		$S \longleftarrow \emptyset $ //o conjunto solução $S$ está vazio inicialmente \\
%        $C \longleftarrow E $ //inicializa o conjunto $C$ com o conjunto inicial $E$ \\
%     \While {$C \neq \emptyset \wedge \neg solucao(S)$ }{
%        $ x \longleftarrow seleciona(C)$ //seleciona o proximo candidato \\
%        $C \longleftarrow C - x$ //remove x do conjunto de candidatos \\
%        \eIf{Viavel($S + x$)}{
%        	$S \longleftarrow S + x$  //adiciona o candidato ao conjunto solucao\\
%        }{$S \longleftarrow S$} //nao atualiza o conjunto solucao \\
     %}
%     \Retorna{S} //melhor solução segundo o critério guloso \\
%   }
% \caption{Algoritmo guloso  \label{guloso}}
%\end{algorithm}


%\subsection{Greedy Randomized Adaptative Search Procedure}
% \
% \linebreak[4]

%Para descrever o \ac{GRASP} será levado em consideração o problema de otimização combinatorial para minimizar a função $f(s)$ para cada solução $S \in X$, sendo definido a partir de um conjunto inicial finito $E = \{ e_{1}, e_{2}, ..., e_{n} \}$ para um conjunto viável de um conjunto soluções viáveis $X \subseteq 2^E$ e pela função objetiva $f:2^E\rightarrow$. 
%O conjunto de soluções viáveis para o problema é definido a partir do E, a função objetivo e das restrições. 
%O \ac{GRASP} é chamado de adaptativo porque os benefícios associados com cada elemento é atualizado a cada iteração da fase construtiva para refletir as mudanças trazidas pela seleção do elemento anterior.~\citeonline{resende:2014}.

%O \ac{GRASP} é definido por como uma heurística gulosa adaptativa de randomização iterativa constituída por duas fases: Uma fase construtiva e outra fase de busca local~\cite{feo:1995} e~\cite{resende:2014}. 
%Na fase de construção é elaborada uma solução, se esta solução não for viável, poderá ser descartada ou é aplicada uma heurística reparatória para alcançar a viabilidade. Uma vez que a solução viável é obtida, a vizinhança que possui esta solução é investigada até que seja encontrado o mínimo local através da etapa de busca local~\cite{resende:2014}.

% Para descrever o \ac{GRASP} \citeonline{resende:2014} leva em consideração o problema de otimização combinatorial para minimizar a função $f(s)$ para cada solução $S \in X$, sendo definido a partir de um conjunto solução finito $E = \{ e1, e2, ..., en \}$ para um conjunto viável de um conjunto soluções viáveis $X \subseteq 2^E$ e pela função objetiva $f:2^E\rightarrowR$. 
% O conjunto de soluções viáveis para o problema é definido a partir do E, a função objetivo e das restrições~\cite{feo:1995}.

%Para a elaboração da etapa construtiva, na qual solução viável é construída iterativamente, um elemento por vez e a cada iteração a escolha do próximo elemento que será adicionado é determinado pela ordem de todos os elementos candidatos em uma lista que respeita a função gulosa, são utilizados algoritmos randomizados.
%Os algoritmos randomizados são importantes para gerar o espaço de busca inicial em uma vizinhança, quebrar loops, habilitar trajetórias diferentes para seguir em uma mesma solução inicial ou simplificar diferentes partes em uma vizinhança extensa~\cite{resende:2014}. 

%Na fase de busca local, é utilizado um algoritmo de busca local que funciona de maneira iterativa, substituindo sucessivamente a solução corrente pela melhor solução encontrada na vizinhança, determinando quando a melhor solução não é encontrada na vizinhança.

%O algoritmo~\ref{grasp} ilustra o funcionamento do \ac{GRASP}, no qual recebe um conjunto de valores candidatos, o máximo de iterações e é utilizado um valor inicial pseudo-aleatório chamado semente e retorna a melhor solução encontrada.

%\begin{algorithm}[h]
% \Entrada{$C, m, s$} //conjunto de candidatos C, maximo de iteracoes m, e semente s \\
% \Saida{melhor solução}
%   \Inicio{
%		\For{$k \longleftarrow 1$ \KwTo $m$ }{
%        	$solucao \longleftarrow  %solucaoConstrutiva(C, s)$ \\
%            $solucao \longleftarrow  %buscaLocal(solucao, melhorSolucao)$\\
%        }
	
%     \Retorna{melhorSolucao} //melhor solução %segundo o critério guloso \\
%   }
% \caption{GRASP \label{grasp}}
%\end{algorithm}

%No contexto de atendimento do \ac{HHCSP} a heurística gulosa foi utilizada por alguns pesquisadores como~\cite{yuan:2015} e~\cite{bard:2012}, que utilizou o \ac{GRASP} para solucionar o \textit{Therapist Routing and Scheduling Problem}, problema semelhante ao \ac{HHCRSP}.

%\subsection{Adaptative Iterated Construction Search}
%não tem artigos falando sobre

%O \ac{AICS}, assim como o~\ac{GRASP}, é um algoritmo utilizado para resolver problemas de otimização, pertencente à sub-área de otimização cujo o conjunto de soluções viáveis é ou pode ser reduzido a um conjunto discreto. 
%Esse algoritmo utiliza a experiência adquirida em soluções passadas para gerar soluções melhores. 
%Uma forma de implementar esta ideia é associar pesos com possíveis decisões que serão tomadas durante o processo construtivo. Estes pesos são adaptados através de múltiplas iterações do processo de busca para refletir a experiência de iterações passadas~\cite{hoss:2005}.

%O funcionamento do \ac{AICS} é dividido em três fases: Construtiva, busca local e adaptação dos pesos. 
%Na fase construtiva, um processo de busca é utilizado para gerar uma solução candidata $s$, em seguida. 
%Na fase de busca local é realizada uma perturbação em $s$, produzindo a solução ótima local $s'$. 
%Por fim, os pesos são adaptados baseados nos componentes da solução utilizados em $s'$ e na qualidade da solução $s'$.
%O processo de busca construtiva utiliza o peso $w$ e alguma função heurística $h$ sobre os componentes da solução selecionados de componentes probabilisticamente selecionados para ampliar a solução parcial candidata atual. 
%Na fase de ajuste de pesos, é realizado um aumento dos pesos que correspondem aos componentes da solução em $s'$, podendo ser utilizados do histórico de busca~\cite{hoss:2005}.

%\subsection{Dynamic Local Search}

%A ideia geral do \ac{DLS} é buscar através do espaço de soluções viáveis e melhorar o retorno da função objetivo, realizando movimentos de subida para escapar de mínimos locais. Uma vantagem deste método é a rápia convergência em direção ao ótimo global da função objetivo sem utilizar derivadas de gradiente ou de ordem superior~\cite{amin:1997}.

%A ideia básica do algoritmo é explorar através do espaço local e global simultaneamente, permitindo a exploração detalhada de áreas de ótimos locais escapando destas.

%\subsection{Busca Tabu}
	
%A Busca Tabu é uma meta-heurística utilizada para guiar uma busca local na exploração do espaço solução além do ótimo local, sendo considerado um método cujo principal objetivo é minimizar a função objetivo $f(x)$~\cite{marti:2013}. A busca tabu é importante para resolver problemas de otimização utilizando uma estrutura de memória flexível~\cite{glover:1990}.
  
%Na busca tabu, é realizada uma busca na estrutura de vizinhança $N$, na qual a solução $x \in N(x)$ encontrada é trocada por uma solução melhor $x' \in N(x)$ atingida a partir de uma operação chamada movimento. 
%Este método permite apenas movimentos que melhorem o valor da função objetivo atual e a execução do algoritmo termina quando não são mais encontradas soluções melhores~\cite{marti:2013}. 
% O autor afirma que Uma falha existente no método decrescente é que o ótimo local, na maioria dos casos, não será o ótimo global, ou seja, geralmente não irá minimizar $f(x)$ em todas as soluções $x$ pertencentes a $X$.

%Na Busca Tabu são utilizados atributos flexíveis baseados em estruturas de memória que permitem a exploração do critério de avaliação e de informações de busca armazenadas; um mecanismo de controle associado, para empregar a estrutura de memória, com base na interação entre condições que restringem e liberam o processo de busca; e a incorporação das funções de memória de diferentes intervalos de tempo de curto a longo prazo, para implementar estratégias para intensificar e diversificar a busca.
%A memória de curto prazo, atribuída a Busca Tabu, constitui uma forma ativa de procurar o melhor movimento possível, sujeito a exigir escolhas disponíveis para satisfazer certas restrições. Estas restrições, que incorporam o limite tabu, são designadas para prevenir  movimentos reversos ou repetitivos e tem como objetivo principal das permitir que o método explore além do ótimo local enquanto mantém a qualidade dos movimentos em cada etapa. 
%Em várias aplicações a memória de curto prazo produz uma solução superior às encontradas com outros processos alternativos. Porém, a memória de longo prazo é importante para obter resultados para problemas mais difíceis. As memórias intermediárias e de longo prazo operam primeiramente como bases estratégicas para intensificar e diversificar a busca~\cite{glover:1990}.

%\subsection{Iterated Local Search}

%O ILS é um método utilizado para resolver problemas de otimização baseado em duas etapas que busca escapar de ótimos locais e alcançar a solução ótima global de forma mais eficiente possível.
%Para encontrar a solução candidata inicial, o ILS obtém a solução ótima local a partir da aplicação do procedimento de busca local. Cada iteração do algoritmo é composta por três estágios: Primeiro é aplicada uma pertubação à solução candidata atual $s$, gerando a solução candidata modificada $s'$; na segunda etapa é executada uma busca local auxiliar em $s'$, obtendo o ótimo local $s''$; na ultima etapa, um critério de aceitação é utilizado para selecionar a solução ótima dentre as soluções $s$ e $s''$~\cite{hoss:2005}. 

%\subsection{Branch and Bound}
  
%O funcionamento do \textit{Branch and Bound} é iniciado a partir de uma pesquisa em todo o espaço solução em busca da melhor solução para o problema.
%Cada iteração possui três componentes principais: O nó selecionado no processo, o calculo do limite (\textit{Bound}), e o ramo (\textit{Branch})~\cite{jean:1999}. 

%Iniciantemente o método \textit{Branch and Bound} considera um ponto qualquer no espaço de busca inexplorado onde será encontrado a melhor solução, este ponto inicial é chamado de conjunto raiz e a melhor solução é definida como infinito. Os subespaços inexplorados são representados por nós gerados dinamicamente e a cada iteração do \textit{Branch and Bound} é processado um nó.
%Em sequência é selecionado o próximo nó. Se a seleção do próximo nó é baseado no valor limite do subproblema, a primeira operação após escolher o nó é a criação de um novo ramo, subdividindo o subproblema em uma determinada quantidade de subespaços que serão investigados na próxima iteração. Para cada subespaço gerado é encontrado uma solução que é compara com a melhor solução atual, mantendo sempre a melhor solução.Caso seja determinado que determinado subespaço não possui a solução ótima, todo o subespaço é descartado~\cite{jean:1999}.

%\subsection{Constraint Programming}

%O problema de satisfação de restrição, consiste em um conjunto finito de variáveis, sendo que cada variável está associada a um valor e a um conjunto de restrições rígidas e flexíveis. A solução para o problema de satisfação por restrição é a satisfação de todas as restrições do domínio~\cite{maria:2008}.

% A diferença entre as restrições flexíveis e não flexíveis é o fato de que, se tratando das restrições flexíveis, o dano causado por sua violação é baixo ou nulo e as restrições não flexíveis devem ser satisfeitas obrigatoriamente \citeonline{blochiger:2003}.

%Existem três modelos computacionais baseados em restrições: Programação por restrições lógicas (\textit{Constraint Logic Programin}), programação restrita contínua( \textit{current Constraint Programming}), e programação restrita contínua pi-cálculo (\textit{concurrent Constraint Pi-calculus}).

%A programação por restrições lógicas é uma extensão da programação lógica, sendo ambas consideradas pertencentes ao paradigma declarativo, dessa forma o programador foca em o que computar ao invés de como computar. Esse método possui um conjunto finito de regras cujo corpo contem conjunções de literais, simbolo atômicos e restrições de domínio\cite{maria:2008}.

%A programação restrita contínua, conhecida como \textit{Current Constraint Programming}, possui a noção de armazenamento, representando o estado do sistema.
%Neste modelo o armazenamento é uma restrição que especifica uma informação parcial sobre possíveis valores de variáveis em qualquer estágio da computação. 
%A programação restrita contínua pi-cálculo,  \textit{Conncurrent Constraint Pi-calculus}, é um mecanismo síncrono e simétrico de interação entre quem envia e quem recebe os dados \cite{maria:2008}.

%A programação restrita contínua pi-cálculo (\textit{Conncurrent Constraint Pi-calculus}) é um mecanismo síncrono e simétrico de interação entre quem envia e quem recebe os dados.

%No contexto do \ac{HHCSP} vários autores como utilizaram o método \textit{Constraint Programming}, mais especificamente, programação lógica por restrições para encontrar boas soluções para o \ac{HHCSP} existindo algumas aplicações em casos particulares, como~\cite{bachouch:2010} e~\cite{cattafi:2012} que aplicaram a suas respectivas cidades.

%\subsection{Variable Neighborhood Search}

%O\ac{VNS} VSN explora uma vizinhança de forma a chegar cada vez mais distante da solução atual, trocando a solução atual por uma nova, se e somente se, a melhoria já foi feita, dessa forma, características de uma solução cuja varias variáveis já possuem seu valor ótimo, serão mantidas e usadas para obter uma boa vizinhança~\cite{hansen:2001}.

%Seja $N_{k} = (1, 2, ..., k_{max})$ um conjunto finito de estruturas de vizinhanças pré selecionadas, $N_{k}(x)$ um conjunto solução da vizinhança $k$ de $x$. A condição de parada é a máxima capacidade da CPU, o maior número de iterações, ou a maior quantidade de iterações entre duas melhorias. Geralmente, sucessivas vizinhanças $N_{k}$ estão aninhadas~\cite{hansen:2001}.

%O funcionamento do VNS é como segue: 
%Na fase de inicialização é selecionado um conjunto de estruturas de vizinhanças $N_{k} = (1, 2, ..., k_{max})$ que será utilizado na busca; escolhido a solução inicial $x$ e a condição de parada.
%Em seguida repete-se até a condição de parada os seguintes passos:
%1 - O conjunto $k$ é inicializado com $1$;
%2 - Até que $k = k_{max}$, repete-se os seguintes passos:
%    a. Gera um ponto $x'$ aleatoriamente em qualquer das vizinhanças de $x$
%    b. Aplica-se algum algoritmo de busca local com $x'$ como solução inicial
%    c. Se o ótimo local for melhor que a solução atual, então move $x <- x''$, continua a busca em $N_{1}(k<-1)$, caso contrário $k <- k+1$;

%\subsection{Variable Depth Search}

%O VDS é um método de melhoria iterativa no qual as etapas de busca local são variáveis sequências de passos de busca simples em uma pequena vizinhança. As restrições nas sequências viáveis em passos simples auxiliam a manter a complexidade do algoritmos baixa~\cite{hoss:2005}.

%O funcionamento do VDS inicia-se a partir da criação de uma lista inicial, escolhida a partir da utilização do método \textit{Randomized Greedy Assignemnt Method}. A cada troca que deve ser realizada, é escolhido o item que possui o menor ganho em penalidade, Para fornecer diferentes soluções iniciais e permitir que a busca também seja usada com reinícios aleatórios, o conjunto de turnos que serão atribuídos é embaralhado. Uma vez que a lista inicial é criada, é possível prosseguir com o \ac{VDS}~\cite{burke:2013}.

% \section{Considerações}

% A partir da análise das heurísticas utilizadas pelos autores pesquisados para resolver o problema de roteamento e escalonamento de profissionais da área da saúde, pode-se concluir que foram utilizados diversos métodos heurísticos para encontrar boas soluções para o problema, como Busca local, técnicas de Inteligência Artificial, Análise de Vizinhança, Programação por Restrição e  Algoritmos Gulosos.
\xchapter{Proposta do trabalho}
{ }%trata-se  da  apresentação  do  estado  em   que  se encontra  o  projeto, dissertando quais são as contribuições esperadas, seguido pelo cronograma de atividades}

% \presetkeys%
%      {todonotes}%
%     {inline,backgroundcolor=yellow}{}

Tendo como estudo de caso o \ac{SAD}, sob a administração do \ac{HHCP} na cidade de Salvador, Bahia, este projeto tem como principal objetivo estudar e analisar a utilização da técnica de Programação por Restrições e propor um modelo para ser aplicada na prática com o intuito de aumentar a quantidade de pacientes atendidos pelo programa e reduzir o tempo gasto pelas equipes de enfermeiras por meio de construção de rotas de visitas mais eficientes. Como contribuições concretas deste trabalho podemos listar:

\begin{enumerate}
\item Elaboração de uma Revisão sistemática de literatura, permitindo uma discussão sobre métodos existentes aplicadas ao \ac{SAD} e auxiliando no processo de obtenção de materiais para pesquisas futuras;
\item Metodologia detalhada, garantindo a reprodutibilidade em trabalhos futuros;
\item Criação de ferramenta prática para o problema do \ac{FESFSUS} que pode ser adaptada para utilização em outros casos de \ac{SAD};
\end{enumerate}

%O \ac{SAD} caracteriza-se como uma modalidade de atenção à saúde composta por um conjunto de ações de prevenção, reabilitação e tratamento de doenças prestadas em domicílio.
%Esse serviço tem se tornado cada vez mais presente como ação de saúde complementar ou substituto à internação hospitalar, pois oferece uma nova modalidade de atendimento às pessoas com quadro clinico estável que necessitam de cuidados.
%Essa modalidade permite maior comodidade aos pacientes, aumentando o conforto e facilitando o apoio familiar, além de auxiliar a reduzir os riscos contaminação hospitalar e reduzir a lotação nos hospitais. 
%A \ac{FESFSUS} é um órgão público, sem fins lucrativos que tem como uma das suas atribuições oferecer serviço de atenção domiciliar a pacientes com médio ou alto grau de complexidade.
%O roteamento e escalonamento da equipe de internação domiciliar ainda é realizado de forma manual no Brasil e em diversos países, por vezes utilizando mais tempo do que o esperado na tarefa de elaborar o escalonamento e o roteamento das equipes e em alguns casos gerando resultados ineficazes.
%Acredita-se que a partir da utilização de heurísticas para a elaboração de escalas de trabalho, e das rotas, serão gerados resultados mais eficientes, e como consequência será possível aumentar a cobertura do programa, assim como sua visibilidade, permitindo a expansão do atendimento a pacientes com baixa complexidade e o aumento o total de pacientes de média ou alta complexidade atendidos.  

%\section{Resultados esperados}

%Como resultados esperados após o fim da pesquisa, temos:

%\begin{itemize}
%\item Revisão sistemática de literatura, permitindo uma discussão sobre heurísticas aplicadas ao \ac{SAD} e auxiliando no processo de obtenção de materiais para pesquisas futuras;
%\item Metodologia detalhada do experimento, garantindo a reprodutibilidade em trabalhos futuros;
%\item Solução heurística para o problema do \ac{FESFSUS} que pode ser adaptada para utilização em outros casos específicos e utilizada em casos gerais;
%\end{itemize}

\section{Metodologia e Métodos}

A metodologia utilizada neste projeto levará em consideração a aplicação da técnica de Programação por Restrições. A Programação por Restrições é uma técnica poderosa para resolver problemas de otimização combinatória que se baseia em uma ampla gama de técnicas de inteligência artificial e pesquisa operacional. A Programação por Restrições é atualmente aplicada com sucesso em vários domínios, como em problemas de escalonamento, planejamento, e roteamento de veículos. A ideia básica da Programação por Restrições é a possibilidade de expressar o problema em forma de restrições lógicas e utilizar um solucionador de restrições de propósito geral para resolvê-las.


A definição clássica de um Problema de Satisfação de Restrição (PSR) pode ser expressado da seguinte forma. Um PSR $\mathcal{P}$ é um tripla $P = (X D, C)$ onde $X $é uma $n$-upla das variáveis $X = (x_1, x_2, ..., x_n)$, $D$ é uma $n$-upla de domínios $D = (D_1, D_2, ..., D_n)$ tal que $x_i \in D_i$, $C$ é uma $t$-upla de restrições $C = (C_1, C_2,. . . , C_t)$. Uma restrição $C_j$ é um par $(R_{S_j}, S_j)$ onde $R_{S_j}$ é um subconjunto do produto cartesiano dos domínios das variáveis em $S_j$. Uma solução para o PSR $\mathcal{P}$ é uma $n$-upla $A = (a_1, a_2, ...,a_n)$ e onde $a_i \in D_i$ e cada $C_j$ são satisfeitos.

% Programação por Restrições consiste em um conjunto finito de restrições contendo conjunções de literais, objetos atômicos comuns sem símbolos de função e restrições sobre um determinado domínio. A Programação por Restrições Lógicas é uma extensão da Programação Lógica, sendo ambas consideradas pertencentes ao paradigma declarativo, dessa forma o programador foca em o que computar ao invés de como computar~\cite{maria:2008}.

%textbf{As restrições lineares} denotam restrições construídas a partir de variáveis
%ujo domínio é dado pelo conjunto dos números reais. Para este tipo de
%estrições têm sido implementados meta-interpretadores de restrições bastante
%ficientes que utilizam o algoritmo Simplex como ponto de partida.

As restrições no conjunto $C$ podem ser classificadas em fortes ou fracas. As restrições fortes são aquelas que devem ser satisfeitas obrigatoriamente; e as restrições fracas formam um conjunto de restrições que devem ser satisfeitas se possível, mas para as quais sabem-se que nem todas poderão ser atendidas.

A partir da análise dos mateiras estudados, foi verificado que o método de Programação por Restrições tem se mostrado eficiente na solução de problemas de escalonamento e de roteamento na áreas da saúde. 

O \ac{SAD} no \ac{FESFSUS} precisa lidar diariamente com as seguintes restrições:

\textbf{Restrições fortes:}
\begin{itemize}
\item Uma enfermeira não pode trabalhar mais de um turno por dia;
\item A demanda de enfermeiras para cada turno deve ser satisfeita durante todo o planejamento;
\item Alguns profissionais de saúde não podem trabalhar simultaneamente.
\end{itemize}
 
\textbf{Restrições flexíveis:}
\begin{itemize}
\item Existe uma limitação para o número de turnos atribuídos as enfermeiras; 
\item Existe uma limitação para a quantidade de dias de folga consecutivos; 
\item Existe um máximo número de dias de trabalho consecutivos;
\item Existe um máximo número de fins-de-semana trabalhados consecutivos;
\item Deve haver um número de dias de folga após uma série de turnos noturnos;
\item Existe um máximo número de fins-de-semana trabalhados em um período de quatro semanas; 
\item Devem ser atribuídos turnos idênticos no fim-de-semana;
\item Uma enfermeira não pode ser atribuída à um turno que demanda mais qualidades do que a mesma possui;
\item Um paciente deve ser atendido sempre pela mesma enfermeira.
\end{itemize}

% A Linguagem de Programação Lógica baseia-se em diversos domínios, destacando-se as restrições booleanas, sobre domínios finitos, sobre intervalos reais e os termos lineares. Outros exemplos incluem listas, conjuntos finitos e árvores.

% \textbf{As restrições booleanas} são tratadas por meta-interpretadores de restrições
% especializados, podendo, no entanto, ser tratadas como um caso particular das
% restrições associadas a domínios finitos para esse problema. Neste último caso
% as variáveis apenas podem tomar dois valores inteiros: 0 (falso) ou 1
% (verdadeiro);

% \textbf{As restrições sobre domínios finitos} são utilizadas em muitas áreas do
% conhecimento. Para satisfação destas restrições usa-se uma combinação de
% técnicas para a preservação de consistência, propagação de valores e pesquisa
% com retrocesso. Cada variável possui associado um conjunto finito de valores
% inteiros, ao qual é dado o nome de domínio da variável. Os valores do domínio
% que levam a inconsistências são removidos do domínio das variáveis durante a
% fase de propagação, enquanto que pela pesquisa se tenta instanciar cada
% variável do problema;

% \textbf{As restrições sobre intervalos reais} são o equivalente das consideradas para os domínios finitos, só que aqui trabalha-se com valores reais em vez de
% valores inteiros. As técnicas de remoção de inconsistências são similares às
% técnicas usadas para os domínios finitos, ou então são baseadas em técnicas
% matemáticas de diferenciação automática ou as séries de Taylor;

% \textbf{As restrições lineares} denotam restrições construídas a partir de variáveis cujo domínio é dado pelo conjunto dos números reais. Para este tipo de
% restrições têm sido implementados meta-interpretadores de restrições bastante
% eficientes que utilizam o algoritmo Simplex como ponto de partida.

\section{Atividades previstas}
%trata-se   da   explicitação   do   modo   como  será desenvolvido o trabalho
Como foi mencionado anteriormente, foi levantada a hipótese de que é possível desenvolver um modelo de programação por restrições para o \ac{SAD} aplicando ao caso específico da equipe de atendimento domiciliar FESFSUS, e assim reduzir o tempo das equipes dentro dos veículos e aumentar a quantidade de atendimentos.

Para validar a nossa hipótese utilizaremos um estudo de caso com o Serviço de Atendimento Domiciliar em Salvador administrado pela \ac{FESFSUS}. Foi realizada uma revisão sistemática de literatura para levantar quais abordagens heurísticas que estão sendo utilizadas atualmente para solucionar o problema de escalonamento e roteamento do \ac{SAD}, e dessa forma, a partir do entendimento de casos já estudados tratar o caso da \ac{FESFSUS} e posteriormente casos mais gerais.

Após o estudo de diversas abordagens levantadas e elaboração da abordagem heurística, serão utilizados dados do próprio projeto estudado e outras bases de dados disponíveis na literatura para fins de comparações experimentais.

\subsection{Cronograma de atividades}

O cronograma seguinte apresenta o desenvolvimento das atividades executadas desde o início da pós-graduação, tendo as atividades representadas a cada mês levado em consideração o período de 24 meses, com início em Novembro de 2016 e final previsto para Outubro de 2018, no qual as disciplinas obrigatórias e optativas, assim como o estágio docente orientado foram realizados no primeiro ano de mestrado, como pode ser visto na Figura \ref{cronograma_atividade}.

%cronograma de atividades
\begin{figure}[H]
\includegraphics[width=1 \textwidth]{cronograma_atividades.png}
\begin{center}
\caption{Cronograma de atividades previstas. \label{cronograma_atividade}}
%Fonte: Elaborada pelo autor
\end{center}
\end{figure}




%%
%% Parte pos-textual
%%
\backmatter

% Apendices
% Comente se naoo houver apendices
%\appendix

% Eh aconselhavel criar cada apendice em um arquivo separado, digamos
% "apendice1.tex", "apendice.tex", ... "apendiceM.tex" e depois
% inclui--los com:
% \include{apendice1}
% \include{apendice2}
% ...
% \include{apendiceM}

% Bibliografia
% BibTeXpress em www.cin.ufpe.br/~paguso/bibtexpress
\bibliographystyle{abntex2-alf}
\bibliography{biblio}

%% Fim do documento


\end{document}
