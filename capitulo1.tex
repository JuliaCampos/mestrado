\xchapter{Introdução}{ }

O \ac{SAD}, caracteriza-se como uma modalidade de atenção à saúde composta por um conjunto de ações de prevenção, de reabilitação e de tratamento de doenças prestadas em domicílio.
Esse serviço tem se tornado cada vez mais presente de forma a complementar ou substituir a internação hospitalar, pois oferece uma nova forma de atendimento às pessoas com quadro clinico estável que necessitam de cuidados.
Essa modalidade de atendimento permite maior comodidade aos pacientes, aumentando o conforto e facilitando o apoio familiar, além de reduzir os riscos de contaminação hospitalar e a lotação nos hospitais \cite{Kergosien:2009}.
Por outro lado, o \ac{SAD} também possui algum desafios, tais como: a necessidade do deslocamento do profissional de saúde, o planejamento da escada de trabalho dos profissionais de saúde envolvidos, o aumento de custos para a família, nos casos da necessidade de manter equipamentos elétricos ligados, e a eventual estadia do cuidador ou enfermeiro.%\cite{portaL:2017}.   

Na busca da melhor qualidade de vida da população e na redução de custos hospitalares, o \ac{SAD} tem sido bastante incentivado em diversos países. 
No Brasil, esse serviço teve início na década de 1960, porém seu funcionamento foi regulamentado pelo Sistema Único de Saúde (SUS) na década de 90, a partir da lei n. 8.080, de 19 de Setembro de 1990 \cite{Silva:2010}, chegando a Salvador em 2012, através da \ac{FESFSUS}. 

A \ac{FESFSUS} é um órgão público, sem fins lucrativos, que atua em 69 municípios do Estado da Bahia desde a Lei Complementar Estadual n. 29, de 21/12/2007, tendo iniciado suas atividades em 2009, começando a atuar na Bahia a partir de 16 de Abril de 2012. A fundação possui como uma das suas atribuições fornecer atenção domiciliar, de forma gratuita para os moradores da cidade de Salvador e regiões metropolitanas. Atualmente existem 9 bases e 135 pacientes internados em domicílio, e uma equipe de profissionais composta por: dois médicos, um enfermeiro, quatro técnicos de enfermagem, e um fisioterapeuta, contando também com profissionais de apoio, sendo eles: um assistente social, um nutricionista e um fonoaudiólogo.

A \ac{FESFSUS} fornece serviço de atenção domiciliar a pacientes com médio ou alto grau de complexidade, como por exemplo: pacientes com sequelas de acidente vascular cerebral, cardiopatas, portadores de paralisia infantil, politraumatizados, perfurados por armas de fogo e pacientes em tratamento oncológico.

Apesar do \ac{SAD} já existir há bastante tempo e em diversos países, ainda existem alguns desafios, tais como, o planejamento do escalonamento das equipes de atenção domiciliar e do roteamento dos veículos destinados a conduzir as equipes que irão realizar os atendimentos.

O Problema de Escalonamento e Roteamento de Equipes do Serviço de Atenção Domiciliar, conhecido como \ac{HHCP}, tem como objetivo determinar como as visitas podem ser agendadas, e como as equipes devem ser compostas, de forma a fazer o melhor uso das equipes de profissionais de saúde e atender os pacientes da melhor forma possível~\cite{Bertels:2006} e~\cite{Decerle:2016}.

% O Problema de Escalonamento de Equipe de Atenção Domiciliar, conhecido como \ac{HHCSP} tem como objetivo reduzir os custos da equipe do \ac{SAD}, de forma que o atendimento seja realizado de forma eficiente, sem prejudicar a qualidade do serviço, para  que isso seja possível, é necessário levar em consideração o tempo de atendimento $T[e_{i}, l_{i}]$ e o fato do atendimento ser realizado por um grupo de profissionais com diferentes habilidades, pela preferência dos clientes e pelo meio de transporte utilizado~\cite{Bertels:2006}. 

% O  Problema integrado de Escalonamento e Roteamento da Equipe de Atenção Domiciliar, o \ac{HHCRSP}, tem como objetivo construir de forma integrada o roteamento dos veículos da equipe de atendimento domiciliar, assim como, o escalonamento das equipes que serão trasportadas em cada veículo, para que seja possível percorrer todos os locais de visita e atender a um conjunto de pacientes de forma eficiente \cite{Decerle:2016}.  

Foi verificado na literatura estudada a existência de diversas abordagens heurísticas, técnicas baseadas em inteligência artificial, e técnicas baseadas em métodos exatos para solucionar o problema citado.

\section{Motivação e justificativa}

O roteamento e escalonamento das equipes de internação domiciliar ainda é realizado de forma manual em diversos países, inclusive no Brasil, tornando o processo ineficiente e muitas vezes gerando resultados insatisfatórios~\cite{cheng:98},~\cite{bachouch:2010},~\cite{tozlu:2016} e~\cite{cattafi:2012}.
Estima-se que o profissional de atendimento domiciliar passam entre $18\%$ a $26\%$ do tempo de trabalho dentro do veículo realizando translados entre os pontos de atendimento~\cite{holm:2014}.

Acredita-se que a partir da elaboração de escalas de trabalho e de rotas mais eficientes será possível aumentar a cobertura do serviço e sua visibilidade, permitindo a expansão do atendimento a pacientes com baixa complexidade e o aumento da quantidade de atendimentos a pacientes de média ou alta complexidade.  

Observando as dificuldades encontradas por diversos pesquisadores no momento de realizar o roteamento e o escalonamento do Serviço de Atendimento Domiciliar em vários países do mundo, foi idealizada uma proposta de elaborar um estudo de caso do \ac{SAD} em Salvador, e desenvolver uma solução heurística com o objetivo de aumentar o número de atendimentos da equipe de internação domiciliar. 

% \section{Metodologia}
% A metodologia utilizada neste projeto levará em consideração abordagens heurísticas e técnicas de teoria dos grafos, além de um estudo de caso e pesquisa qualitativa e quantitativa.

\section{Hipótese e objetivos}

Nesta seção será apresentada a hipótese do problema, assim como o objetivo geral e específicos.

\textbf{Hipótese}: \emph{É possível desenvolver uma heurística para o \ac{HHCP} aplicando ao caso específico da equipe de atendimento domiciliar FESFSUS, em Salvador, dessa forma, aumentando produtividade da equipe a partir da redução do tempo dentro do veículo, e auxiliando no aumento da eficiência do atendimento domiciliar em Salvador e região metropolitana,  e contribuindo com a expansão da cobertura do programa, possibilitando o atendimento a pacientes com baixa complexidade.}

Este trabalho tem como objetivo principal desenvolver uma solução heurística para maximizar o número de atendimentos da equipe de atenção domiciliar e aplicar ao projeto FESFSUS em Salvador. 

Buscando alcançar o objetivo principal, temos os seguintes objetivos específicos:
\begin{itemize}
\item Analisar o tempo utilizado pela equipe do \ac{SAD} no percurso entre pontos de atendimentos;
\item Analisar o escalonamento das equipes que serão transportadas em cada veículo;
\item Analisar as rotas de veículos elaboradas pela equipe do \ac{SAD};
\item Analisar heurísticas existentes para os problemas de roteamento e escalonamento do \ac{SAD};
\item Propor uma heurística para o problema de escalonamento e roteamento de equipes do \ac{SAD};
%\item Minimizar os custos do \ac{SAD}
%\item Propor uma heurística para o problema integrado de escalonamento e roteamento de veículos do \ac{SAD};
\end{itemize}


\section{Organização do trabalho}
Este trabalho está organizado da seguinte forma: No capítulo 2 serão apresentados os problemas clássicos de roteamento e de escalonamento; no capítulo 3 serão apresentados problemas de roteamento e escalonamento aplicados à área da saúde; no capítulo 4 é apresenta uma revisão sistemática de literatura, descrevendo as heurísticas existentes; e por fim, no capítulo 5 são apresentados os trabalhos relacionados a proposta da dissertação. 
    
% \begin{figure}[ht]
% \begin{center}
% \begin{tikzpicture}[scale=0.4]
% 	%ponto central
% 	\draw node[draw] at (0, 0) {$0$};
% 	\draw node[draw] at (5,5) {$N+1$};

%     %lado direito x
%     \draw[->, red, dotted, thick] (1.5, -0.5) node[below] {$c_{0,1}$} (0.2, 0) -- (2.8, -1.0) ;
%    	\fill[black] (3,-1) circle (2mm) node[above right] {$v_1$};

%     \draw[->, red, dotted, thick] (4.4, -0.6) node[below] {$c_{1,2}$}  (3.2, -1) -- (5.8, 0);
%     \fill[black] (6,0) circle (2mm) node[above right] {$v_2$};

%     \draw[->, red, dotted, thick] (7.4, -0.1) node[below] {$c_{2,3}$}  (6.2, 0) -- (8.8, 0);
%     \fill[black] (9,0) circle (2mm) node[above right] {$v_3$};   
 
%     \draw[->, red, dotted, thick] (10.4, -0.1) node[below] {$c_{3,4}$}  (9.2, 0) -- (11.8, 0);
%     \fill[black] (12,0) circle (2mm) node[above right] {$v_4$};    

%     \draw[->, red, dotted, thick] (10.8, 2) node[below] {$c_{4,5}$}   (11.8, 0.1) -- (9.3, 3);
%     \fill[black] (9.1, 3.1) circle (2mm) node[above right] {$v_5$};    

%    \draw[->, red, dotted, thick] (6.9, 3) node[below] {$c_{5,6}$}   (9, 3.1) -- (6,2 );
%    \fill[black] (5.8, 2) circle (2mm) node[above right] {$v_6$};    

% 	\draw[->, red, dotted, thick]  (4.1, 2.5) node[below] {$c_{6,7}$}  (5.6, 2) -- (3,2 );
%    \fill[black] (2.8, 2) circle (2mm) node[above right] {$v_7$};  
  	
% 	\draw[->, red, dotted, thick]  (3.2, 4) node[below] {$c_{7,N+1}$}   (2.7, 2) -- (4.8,4.6 );
% \end{tikzpicture}
% \end{center}
% \caption{Grafo de pesos?}
% \end{figure}
