%%%%%%%%%%%%%%%%%%%%%
% Resumo em Português
%%%%%%%%%%%%%%%%%%%%%

\resumo
O Serviço de Atenção Domiciliar é importante para a redução de custos de hospitais e para melhor qualidade de vida das pessoas que necessitam de cuidados médicos.
Na cidade de Salvador, o Serviço de Atendimento Domiciliar é realizado pela Fundação Estatal Saúde da Família.
O roteamento e escalonamento da equipe de internação domiciliar é realizado de forma manual no Brasil e em diversos países, por vezes gerando resultados ineficazes.
O Problema de Roteamento e Escalonamento de Enfermeiras tem como objetivo elaborar um cronograma de forma que cada profissional de saúde visite um conjunto de pacientes, faça uma pausa e finalize suas atividades dentro da janela de tempo de trabalho.
Esse trabalho tem como objetivo desenvolver uma solução heurística para minimizar os custos e aumentar o número de atendimentos da equipe de atenção domiciliar e aplica-la ao projeto FESFSUS em Salvador.
% Palavras-chave do resumo em Portugues
\begin{keywords}
Escalonamento de Equipes, Roteamento de Veículos, Serviço de Atenção Domiciliar, Heurísticas, Pesquisa Operacional
\end{keywords}

%%%%%%%%%%%%%%%%%%%
% Resumo em Ingles
%%%%%%%%%%%%%%%%%%%

\abstract
The Home Care Service is important for reducing hospital expense and improving the quality of life for people who need medical care.
In the city of Salvador, the Home Care Service is carried out by the State Foundation for Family Health (FESFSUS).
The Nursing Routing and Scheduling Problem aims to elaborate a schedule so that each health professional visits a group of patients, pauses and finishes their activities within the working time window.
This work aims to develop a heuristic solution to minimize costs and increase the number of home care staff and apply it to the FESFSUS project in Salvador.
% Palavras-chave do resumo em Ingles
\begin{keywords}
Crew Scheduling, Vehicle Routing Problem, Home Health Care, Heuristics, Operational Research.
\end{keywords}
