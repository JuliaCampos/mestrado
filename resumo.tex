%%%%%%%%%%%%%%%%%%%%%
% Resumo em Português
%%%%%%%%%%%%%%%%%%%%%

\resumo
O Serviço de Atenção Domiciliar, caracteriza-se como uma modalidade de atenção à saúde composta por um conjunto de ações de prevenção, reabilitação e tratamento de doenças, prestadas em domicílio. Esse serviço tem se tornado cada vez mais presente como ação de saúde complementar ou substituto à internação hospitalar, pois oferece uma nova modalidade de atendimento às pessoas com quadro clinico estável que necessitam de cuidados. O roteamento e escalonamento da equipe de internação domiciliar é realizado de forma manual em diversos países, inclusive no Brasil, tornando o processo ineficiente e muitas vezes gerando resultados insatisfatórios. Estima-se que o profissional de atendimento domiciliar passam entre 18\% a 26\% do tempo de trabalho dentro do veículo realizando translados entre os pontos de atendimento. Na literatura, o \textit{Home Health Care Routing Scheduling Problem} (HHCRSP) tem como objetivo construir de forma integrada o roteamento dos veículos da equipe de atendimento domiciliar, assim como, o escalonamento das equipes que serão trasportadas em cada veículo, para que seja possível percorrer todos os locais de visita e atender a um conjunto de pacientes de forma eficiente. Este trabalho tem como objetivo investigar o HHCP e desenvolver uma solução heurística para aumentar o número de atendimentos da equipe de atenção domiciliar, tendo como estudo de caso a Fundação Estatal de Saúde da Família em Salvador.
% Palavras-chave do resumo em Portugues
\begin{keywords}
Escalonamento de Equipes, Roteamento de Veículos, Serviço de Atenção Domiciliar, Heurísticas, Pesquisa Operacional
\end{keywords}

%%%%%%%%%%%%%%%%%%%
% Resumo em Ingles
%%%%%%%%%%%%%%%%%%%

\abstract
The Home Care Service is characterized as a modality of health care composed of a set of actions for prevention, rehabilitation and treatment of diseases, provided at home. This service has become increasingly present as a complementary health action or substitute for hospital admission, as it offers a new modality of care for people with stable clinical conditions that need care. The routing and scheduling of the home hospitalization team is carried out manually in several countries, including Brazil, making the process inefficient and often generating unsatisfactory results. It is estimated that the home care professional spend between 18 \% to 26 \% of the working time inside the vehicle doing transfers between the service points. In the literature, the Home Health Care Routing Scheduling Problem (HHCRSP) aims to build in an integrated way the routing of the vehicles of the home care team, as well as the scheduling of the teams that will be transported in each vehicle, so that it is possible to cover all the places of visit and to attend a set of patients efficiently. This work aims to investigate the HHCP and develop a heuristic solution to increase the number of home care staff, with the State Foundation for Family Health in Salvador as a case study.
% Palavras-chave do resumo em Ingles
\begin{keywords}
Crew Scheduling, Vehicle Routing Problem, Home Health Care, Heuristics, Operational Research.
\end{keywords}
