\xchapter{Problemas de escalonamento e roteamento na área da saúde}
{ }

\presetkeys%
    {todonotes}%
    {inline,backgroundcolor=yellow}{}
 %\todo{}
%\section{Escalonamento e roteamento de equipes de enfermagem}


\section{Problema de Escalonamento de Enfermeiras}

%No restante do texto utilizaremos o termo enfermeiras como uma nomenclatura genérica a todos os profissionais de saúde que possam estar envolvidos no Serviço de Atendimento Domiciliar e termo cliente para todas as pessoas que recebam qualquer atendimento de um profissional do \ac{SAD}.

O \ac{PEE} consiste em encontrar uma atribuição de enfermeiras a turnos respeitando a um conjunto de restrições \cite{ioanis:2015}. Seja $E = \{e_1, e_2, \ldots, e_{|E|}\}$ um conjunto de enfermeiras, Seja $D = \{d_1, d_2, ..., d_{|D|}\}$ um conjunto de dias, seja $F = \{ f_1, f_2, ..., F_{|F|} \}$ um conjunto de tarefas e seja $B$ um conjunto de turnos, onde cada elemento de $B$ está mapeado da seguinte forma: ($M$, matutino), ($V$, vespertino), ($N$, noturno) e ($-$, folga). Uma solução para o \ac{PEE} pode ser representada por uma matriz $M_{|E|\times |D|}$, na qual cada célula $m_{e_i,d_j} \in M$ contém a atribuição para o turno $t\in B $ que deverá ser comprido pela enfermeira $e_i$ no dia $d_j$.


A Tabela~\ref{enfermeira_dia} apresenta um exemplo de escalonamento de enfermeiras cujos valores $n_1$, $n_2$ e $n_3$ representam os enfermeiras escalonados no período de sete dias sendo representados pelos valores $d_1$, $d_2$, $d_3$,$d_4$, $d_5$, $d_6$, $d_7$.

\begin{table}[ht]
 \centering
\caption{Exemplo de escalonamento de três enfermeiras no horizonte de sete dias de trabalho \label{enfermeira_dia}}
\begin{tabular}{r|l|l|l|l|l|l|l}
  	   & $d_1$ & $d_2$ & $d_3$ & $d_4$ & $d_5$ & $d_6$ & $d_7$ \\ \hline
 $e_1$ & $M$  & $V$  & $N$  & $M$  & $V$  & $V$  & $-$ \\ \hline
 $e_2$ & $V$  & $-$  & $M$  & $N$  & $N$  & $-$  & $V$\\ \hline
 $e_3$ & $M$  & $M$  & $V$  & $N$  & $-$  & $M$  & $V$
\end{tabular}
\end{table}

No contexto deste trabalho, al\'em de resolver o problema de escalonamento de enfermeiras, tamb\'em devemos nos preocupar com a ordem de atendimento dos pacientes. \'E importante que um determinado paciente que iniciou seu tratamento sob os cuidados da enfermeira $e_i$ continue seu tratamento com a mesma. Sendo assim, \'e necess\'ario integrar o problema de escalonamento ao problema de rotamento, o que aumenta o desafio dessa proposta.

\section{O Problema de Escalonamento e Roteamento de Enfermeiras}

O objetido do \ac{PERE} consiste em  encontrar um cronograma, de modo que cada enfermeira visite um conjunto de pacientes, faça uma pausa e finalize as atividades prevista dentro da janela de trabalho. No \ac{PERE}, devemos alocar um conjunto heterogêneo de enfermeiras  que realizam visitas a um conjunto de pacientes  $P = \{ p_1, p_2, \ldots, p_{|P|} \}$, sendo que cada visita realizada a um paciente $p \in P$ por uma enfermeira $e \in E$ possui uma janela de tempo $[c_{p}, f_{p}]$, com $c_{p}$ representando o instante de tempo para que a enfermeira possa iniciar o atendimento e $f_{p}$ representando instante de tempo m\'aximo para a enfermeira iniciar o atendimento ao paciente $p$. A viagem entre as residências de dois pacientes $p_i$ e $p_j$  possui o custo $c_{p_ip_j}$~\cite{rasmussenm:2012}.   

\cite{cheng:98} prop\^os uma modelagem para o \ac{PERE} em um grafo, simples e direcionado $G = (V,A)$. O conjunto de nós $V$, consiste em três conjuntos disjuntos: O conjunto $E$ de enfermeiras, o conjunto $P$ de pacientes e um conjunto $L$ de pausas. O conjunto de arcos $A$, é baseado em uma noção de compatibilidade. Existe uma relação binária entre cada enfermeira e qualquer outro nó. Em outras palavras, a cada enfermeira é permitida ou impedida de visitar um nó em $V$. Cada enfermeira $e \in E$ possui uma janela de tempo $[c_e, f_e]$ que indica o horário de início e fim do seu turno.  Cada paciente $p \in P$ possui o tempo $d_q$, que indica a duraç\~ao estimada para receber o atendimento. O conjunto de pausas consiste em um nó único para cada profissional de saúde onde cada pausa possui uma duração e uma janela de tempo \cite{cheng:98}.

J\'a \cite{mansini:2016} define uma int\^ancia do \ac{PERE} da seguinte forma. Seja $W = \{w_0, w_1, w_2, ..., w_{|W|}\}$ um conjunto de locais de interesse, em que $w_0$ define o ponto de partida do \ac{SAD}, seja $P$ um conjunto de pacientes e seja $T = \{ t_1, t_2, \ldots, t_{|T|}\}$ um conjunto de tarefas.  
Uma inst\^ancia do \ac{PEE} pode ser representada por um grafo ponderado $G = (V, A,\tau)$, no qual $V = \{w_0\} \cup Q$ é um conjunto de vértices sendo que cada vértice, exceto $w_0$, representa um paciente $p\in P$ que deve ser visitado pela enfermeira $e\in E$, o conjunto de arestas $A = \{ (i,j): i \in V, j \in V, i \neq j \}$ representa o translado entre os locais $w_i$ e $w_j$, com tempo de viagem $\tau_{w_iw_j}$. 

Dada a import\^ancia econ\^omica e o impacto social deste tema, v\'arios autores propuseram diversos m\'etodos e modelagens para solucionar o \ac{PERE}. Na pr\'oxima seç\~ao apresentaremos uma revis\~ao sistem\'atica de literatura onde s\~ao listadas algumas refer\^encias bibliogr\'aficas relacionadas ao tema desta proposta de dissertaç\~ao, juntamente com uma an\'alise de suas principais contribuiç\~oes e objetivos.

%O \ac{PERE} consiste em determinar um cronograma que atribui a cada enfermeira $e$ quais clientes $q$ devem ser visitados e qual tarefa $f$ deve ser realizada, respeitando um conjunto de restrições\cite{mansini:2016}.

%Sendo que o gráfico não contém loops nem arcos entre dois nós de enfermeiro ou entre dois nós de almoço.

%O objetivo do \ac{PERE} é maximizar a quantidade de trabalho\footnote{Pense melhor... tem certeza de quem o objetivo seja mesmo maximixar a quantidade de trabalho} minimizando a quantidade de viagens necessária. \cite{cheng:98}\footnote{Verifique novamente no artigo. Certifique-se de que esse artigo nao eh um technical report}

%\section{O Problema de Roteamento de Enfermeiras}

%\subsection{Problema de Escalonamento de Equipe de Serviço de Atenção Domiciliar}

%O \ac{HHCSP} tem como objetivo realizar o escalonamento e roteamento da equipe de atenção domiciliar de forma a gerar resultados eficientes~\cite{bachouch:2010}.
